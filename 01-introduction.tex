
Wikipedia project is one of the tremendous successful project of knowledge
production ever, with more than 3.5 million articles for the English
version and nearly one million visits per day\footnote{For statistic on Wikipedia, visit \url{http://stats.wikimedia.org/EN/},
and for an historical presentation, see \citet{Lih09}.}. This is done by the coordination of thousands of people which give
their time and their knowledge to construct the article, making this
project one of the biggest collective intelligence project ever created\footnote{\citet{Olleros08} proposes a good introduction to the encyclopedia
and how it has innovated in the production of encyclopedic knowledge.}. This volunteering online open projects seem to have found original
answers to \citet{Olson65}'s paradox: without direct monetary retribution,
there are enough non-free riders to make the project work. However
this tremendous success, this project seems to steam, as there is
a growing concern about the difficulty to recruit and retain new editors\footnote{\url{http://en.Wikipedia.org/wiki/Wikihttp://en.Wikipedia.org/wiki/Wikipedia:Areas_for_Reform#Do_we_have_a_problem_recruiting_new.2C_or_retaining_current.2C_editors.3Fpedia:Areas_for_Reform#Do_we_have_a_problem_recruiting_new.2C_or_retaining_current.2C_editors.3F}
and \url{http://meta.wikimedia.org/wiki/Research:Index}.}, problem already stressed by researchers \citep{Ortega09}.

From an Information System research point of view, this example of
open knowledge projects should provide useful information on how structuring
online open knowledge project for it to succeed\footnote{Even if the comparison between different industries must be done very
carefully, as shown by \citet{MullerSeitzReger09} on the comparison
between open-source with Open Source car and Wikipedia projects.}, but also for internal organization's projects, as firms are institutions
created to allow collaboration \citep{Simon57,MarchSimon58}, and
even defined by \citet{Grant96} as \textquotedbl{}institution for
integrating knowledge {[}of its members{]}''. And regarding this
function, the wiki tool, allowing distant and sequential collaboration
around a structured document seems to be very promising \citep{HasanPfaff06a,HasanPfaff06b,HasanPfaff06c}.
It also, for libraries, a new platform to stock information and knowledge
resources, to to promote digital collection and thus to reach new
users \citep{PressleyMcCallum08}, and today a prominent source of
online information, notably for health \citep{LaurentVickers09},
where false information may have dramatic consequences. 

But, to be used as a model, how the model works must be better understood.
This is needed for internal purpose too: the Wikipedia project managers
may want to monitor the activity: is the English Wikipedia more efficient
than the French one, or are projects (portals) more or less efficient,
more or less productive, is there a minimal, an optimal number of
editors for an article\footnote{For research questions pointed by the Wikimedia Foundation, which
support the Wikipedia project, see \url{http://meta.wikimedia.org/wiki/Research:Index}.}?

And, to quote \citet[p. 124]{CrowstonHowisonAnnabi06}, to ''be able
to learn from teams that are working well, we need to have a definition
of \textquoteleft working well\textquoteright ''. To do so, we first
rely on the literature of Information System, but also on the literature
of knowledge common, to propose a framework of how the elements interact.
Then, what we propose here is not a review of all the articles written on Wikipedia, but a review of the existing literature on how Wikipedia works.
