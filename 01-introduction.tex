The competitive advantage of a firm lies in its capacity to manage the transfer of knowledge of individuals and groups within the organization \citep{LocanderNapierScamell79,KogutZander92}. Firms do not create knowledge so much as select and coordinate knowledge \citep{Grant96,TiwanaMcLean05} required for projects \citep{Teeceal97} and by teams \citep{WuchtyJonesUzzi07}. The project manager can thus be viewed as the manager of a temporary organization \citep{TurnerMuller03}, who exercises leadership through stewardship \citep{DavisSchoormanDonaldson97}.

%%%% MISQ fit justification sentences %%%% 
The transitive nature of team members has increased in recent years, especially in software development teams, because 
of the emergence of Free/Libre, Open Source Software projects (FLOSS), where the teams are built on voluntary and 
competence-based criteria \citep{Fitzgerald06}.
Today, for efficiency and flexibility, this structure is copied by firms in two ways:
by outsourcing part of their intellectual production through open innovation or crowdsourcing, which require
internal reorganization \citep{StolFitzgerald14}; or through inner source, where the firm enables voluntary cross-team 
collaboration within the organization in order to internally replicate the benefits of FLOSS methods of production
\citep{Stoletal11}. However, these changes may be difficult for companies to effectively implement, both because of
the company culture and because of the challenge it creates for managers, who may have difficulty accounting for
and evaluating their employees' work.
%%%% end %%%%

In theory, leadership is expressed through selection and motivation of good team members \citep{Guimeraetal05}, and in the integrity and benevolence demonstrated by a manager, which contribute to building trust \citep{KayworthLeidner02,JarvenpaaKnollLeidner98}. It has been shown that personality traits such as agreeableness contribute to individual performance in teaming situations \citep{mount:1998:five}, but it is unclear in what proportion these traits are necessary, and their relative importance compared to technical competencies. The difficulty of establishing this assessment lies in the fact that the optimal mix of qualities varies by industry, company and team \citep{bassellier:2001:information,wingreen:2017:professionals}. There are, to the best of our knowledge, no evaluation methodologies available to describe the attributes desired of an individual in a team, which takes into consideration the situation as well as the manager's preferences and knowledge.

%%%% MISQ fit justification sentences %%%% 
In the context of IS-based organizations and virtual teams, most individual attributes are only observable in virtual
interaction, making some characteristics potentially impossible to detect. At the same time, it must be asked: if they are
not observable, are they important?

Having a methodology which allows the manager to express the characteristics they really look for when assessing their team members would have several potential managerial benefits.
%%%% end %%%%

First, individuals could determine if they were a good match for the project based on how well they aligned with the profile. Alternately, in an assigned team, dynamic team composition could be constructed based on necessary qualities. 

Second, project managers could explicitly identify the qualities they value in their team and use this to steer development. Coaching individuals in areas where improving their skills would most benefit their participation in the team, articulating why certain individuals are problematic, and developing plans for improvement or expulsion are examples of steering enabled by assessment. Finally, a transparent exposition of desired attributes provides the manager with a non-arbitrary way of expressing preferences, leading to greater opportunities for consistency and fairness. 

%%%% MISQ fit justification sentences % %%%%%
The goal of this paper is to address this need for an assessment tool in the case of virtual teaming, where 
the qualities are primarily observed through interaction (globally distributed teams, open and inner source software, 
and other types of online knowledge production).
The methodological questions we want to address here are twofold. First, can we make the managers express a distinct and manageable set of
characteristics which can be used to evaluate a team member? Second, can we establish the proportion of relevance provided by
each characteristic in the process of evaluating a virtual team member?
%%%% end %%%%

Evaluating virtual team members with consideration to multiple attributes while incorporating the perspective of the team manager is clearly a complex decision problem. Multi-Criteria Decision Aiding (MCDA) helps decision makers (DM) make better choices based on their preferences and priorities, especially when, in this case, the set of decision alternatives---in our case team members---can be described via multiple, usually conflicting properties.
%%%% MISQ fit justification sentences % %%%%%
In this work we demonstrate how careful design of a MCDA protocol can lead to new insights for managers evaluating team members.
We also discuss how this can be incorporated into the knowledge production technological infrastructure.
%%%% end %%%%

This is why we see our work as falling within the design science paradigm in IS \citep{Vonetal04,MarchStorey08}, which aims to provide researchers and practitioners with theoretical and methodological tools to better define, model, and analyze a problem, in this case teaming and its alignment with the manager's expectation. Accordingly, this paper follows the framework proposed by \cite{GregorHevner13} in its organization: the next section reviews the theoretical background and develops our hypotheses of the qualities team members are expected to possess and discusses how a manager can evaluate a person using a multi-dimensional perspective using MCDA. In Section 3, we expand on the design of such a methodology, while in Section 4 we present an example of what the application of the method can bring. The application case is the elicitation of six managers' preferences in the case of Free/Libre/Open Source Software (FLOSS) teaming, two of which are presented in extensive detail. We discuss its use and limitations in Section 5 before concluding with an explanation of its relevance to both management researchers and practitioners.
