Actually, speaking of the Wikipedia project can be viewed as a short-path,
as each language proposes a version, and has its own collection of
articles, more or less common with the English version.

% Cultural Environnement

\citet{HechtGergle10} studied 25 language projects and found that
the articles present in all the projects represent only 1\% of the
total, when 74\% of the articles were present in one language only.
For instance, \citet{CallahanHerring11} showed that the famous persons
for the English and the Polish Wikipedia are not the same \footnote{See also the differences in the periods of contribution, where some
language Wikipedias more contributed during the weekdays, such as
the English one, and other during the week-end (Japanese, for instance)
in \citep{YasseriSumiKertesz12}.}. 

\citet{PfeilZaphirisAng06}, analyzing the way French, German, Dutch
and Japanese contributed to the article ''game'', show a correlation
between Hosfstede\textquoteright s cultural dimensions \citep{Hofstede91,HofstedeMcCrae04}
and the way people perform different kinds of actions in the writing
of the article (number of correction, deletion, contributions). For
instance, there are statistically significant more courtesy behaviors
in the large Wikipedias that in the small ones (in terms of number
of articles) and in Eastern Wikipedias than in Westerns ones (\citealp{HaraShachafHew10},
in a comparative study of the English, Hebrew, Japanese, and Malay
Wikipedias). \citet{Stviliaetal09} compared the English Wikipedia
Feature Article quality process with the one of the Arabic and the
Korean Wikipedia. However the small size of the sample for the two
last (91 for the Arabic and 25 for the Korean), they showed that for
almost all the criteria used by the users to evaluate the articles\footnote{Well-written, Comprehensive, Factually accurate, Neutral, Stable,
Consistent with the style guidelines, Images, Appropriate focus and
length for the English and the Arabic, Well-written, Appropriate Length,
Neutral, Accurate, Links, Images for the Korean).}, there are strong variations between the three projects (at the date
of the dumps copy, June 2008). There are strong variation too, in
the representation of the knowledge (see, for instance, the study
by \citealp[part 6, p. 10]{Hammwohner07}, on how categories are subordinate
in various European languages). As early as 2005, \citet{Voss05}
noted a strong variation in the number of edit made by anonymous between
languages Wikipedia (10\% in the Japanese one, and 40\% in the Italian
one at the end of 2004), whereas the number of edit by people distribution
was quite similar. Some projects may have specific difficulties, making
the path of evolution barely comparable to the others, such as the
Chinese Wikipedia, which has had to solve the conflict between different
writing forms \citep{Liao08}, or small number of speakers Wikipedia,
which are quite empty of real articles, as shown by \citet{vanDijk09}.
This author also show the importance of the Internet access, but also
of the number of people able to translate articles from the English
to explain the difference in Wikipedia growth\footnote{On that aspect, he relied on the analysis of the Indonesian Wikipedia
made by \citet{SoekatnoGiri05} }, a result also stressed by \citet{Stviliaetal09}. Finally, as \citet{LiuIyer07}
pointed out, these variations may be also due to variations in age
and scale of the projects. As \citet{MarwellOliver93,OliverMarwellTeixeira85}
explained, in collective projects at the initial stage, people are
few and efforts costly, in the diffusion phase, the number of participants
grows as their efforts are rewarding, but with increasing need for
coordination, and the mature phase, some inefficiency may appear as
the contributors are more numerous than the work needed \citep[note that this has been empirically tested in the case of open online communities by][]{Alluvattietal11}.
if until 2006, and according to Wales, the English Wikipedia was written
by a small group of editors (talk at Stanford University in 2006,
cited in \citealp{Swartz06}), as early as 2006, \citet{Burioletal06}
showed that there were some indications of a permanent regime (they
call ''maturity''): for instance the constance of the average edits
per users, or the ''high correlation between PageRank and indegree,
indicating that the microscopic connectivity of the encyclopedia resembles
its mesoscopic properties'' (p. 8). \citet{Suhetal09} confirmed
this slowdown for the English Wikipedia. \citet{LamRiedl11} confirmed
that the English Wikipedia's production follows a S-shaped curve.

But what these various in projects have in common that healthiness
of a language project depends on the characteristics of the Internet
community (especially the number of Internet users, and the wealthiness
of of the population, according to \citealp{Rask08}), and on the
people's competencies \citep{GlottSchmidtGhosh10}, especially the
number of tertiary educated people within the population \citep{CrowstonJullienOrtega13}.
The global structure of the project, measured as a network, the nodes
being the articles and the links the links between the articles, seems
also to be the same, in terms of ''degree distributions, growth,
topology, reciprocity, clustering, assortativity, path lengths, and
triad significance profiles'', at least for the main projects \citep{Zlaticetat06}.
People also seem to contribute during the same period of time of the
day (between 1pm and 11 pm, still in \citealp{YasseriSumiKertesz12}).
Finally, \citeauthor{ZhaoBishop11}'s Delphi study (2011 p. 725),
points that the factors underlined by Wikipedia researchers to explain
Wikipedia's success are, in addition to its success, the rules in
use (especially the ones which promote communications) and the technical
structure which supports these rules and facilitates the editing.

% Environnement technique

As pointed out by \citet{HessOstrom06b}, as by the actor network
theory \citet{AkrichCallonLatour06,Latour05}, the artifacts, or the
tools used by the online communities are important to understand how
this community can work. Or, to quote\citet[p. 6]{NiederervanDijck10},
\textit{''Wikipedia {[}is{]} a gradually evolving sociotechnical
system that carefully orchestrates all kinds of human and non-human
contributors by implementing managerial hierarchies, protocols and
automated editing systems''}. Two tools seem to be of particular
importance to understand what Wikipedia is: first of all, the program
allowing to edit and manage the contributions, the MediaWiki. Several
structuring features of the Wikipedia collaborative organization are
due to this software \citep{PrasarnphanichWagner09}, such as the
collective editing, but also the existing of a talk page for each
articles, or the way links are made between articles and to the exterior.
This tool suffers certain limitations, from a content management orthodox
point of view, according to \citet{Doyle08}: there ''are clueless
about today\textquoteright s content management best practices like
content reuse, modularity, structured writing, and information typing''.
But as emphasized by \citet{Ciffolilli03}, in a transaction cost
theory based analysis of Wikipedia, ''Wiki technology in a way literally
cancels transaction costs for editing and changing information''.
This is a bit optimistic, as people have to understand how the changes
are stored and still have to discuss to content (see the section \ref{sec:Process,-or-the},
page \pageref{sec:Process,-or-the} on that aspect), but it surely
drop this cost \citep[p. 252]{RafaeliAriel08}, and also drop the
cost of degradation, or ''graffiti attacks'' \citep{Ciffolilli03},
as the tool keeps memory of the former versions and makes it easy
reversing. It also helps people, and especially the editors, in the
organization and in the structuring of their tasks \citep{Sundin11}.

However the importance of this wiki-based technical platform \citep{NiederervanDijck10},
it seems that, as the project has grown up, the socio-technique community
evolved ''from an informal trust-based community with few formal
roles to a socio-technique community where the social mechanisms,
and not the software architecture, supports knowledge management processes''
\citep{Jahnke10}. Even if it seems paradoxical, this is well illustrated
by a second software tool which has gained growing importance with
the success of Wikipedia, the bot. Because, a explained by \citet[end of p. 5 and following]{Geiger11}:
''Bots, like infrastructures in general, simultaneously produce and
rely upon a particular vision of how the world is and ought to be,
a regime of delegation that often sinks into the background {[}...{]}''

% Transition to the second tool, the bots

Bots are responsible for most of the publications of articles in endangered
language Wikipedias (\citealt[p. 12]{NiederervanDijck10} based on
\citealp{Rogersetal08}), resulting that most of these articles are
empty \citep{vanDijk09}. In the same time, still as shown by \citet{NiederervanDijck10}
on the list of the USA towns, the automate creation of articles facilitates
the completion of these articles in the future. The role played by
these automate tools is well illustrated by \citeauthor{GeigerRibes10}'s
analysis (2010) of the vandal fighting, and of the role played by
software in this task: in a comparison with the analysis of ship navigation
by \citet{Hutchins96}, they show how these tools implement human
decision facilitating their execution (the detection of task considered
as vandalism, or the automatic and comprehensive creation of a set
of information) and their management by automating the rules (gradation
in the sanction, formated messages), making these tasks ''mundane''.
But still pointed out by these authors, the definition of vandalize
and the punishments remain a moral choice, and the humans implement
the rules by programing these tools. Thus, these tools also are discussed
\citet[end of p. 5 and following]{Geiger11} ''\textendash{} that
is, until they do not perform as expected and generate intense controversies''.

% Rules in use.

These rules are numerous, increasing in number and complexity \citep[analyzing the English Wikipedia's rules]{Butleretal08},
and ranging from the the more formal and explicit (intellectual property
rights) to the more informal. 

First of all, it must be stressed that, as for software, articles
are protected by copyright laws, and that it is this protection which
grants the producer to license (in the Latin sense, authorize) the
user to use it. Here, this protection is used to ''copyleft'' the
use, to quote Stallman, but it comes also with obligations. The ''Creative
Commons Attribution-ShareAlike'' License, used for Wikipedia, allow
to use, to change, but if redistributed, the work built upon the article
has to be redistributed under the same terms and conditions\footnote{\url{http://creativecommons.org/licenses/by-sa/3.0/}}.
If this protection is juridically efficient is matter of debate (see\citealp{Wielsch10},
on that question), but this frameworks the vision people have about
the project and of its openness. Another legal based characteristics
of Wikipedia is that the name is a registered trademark of the Wikimedia
Foundation (non-profit organization), which also owns the technical
infrastructure which operates the service (servers). So, if this foundation
does not own the content, its own the right to ultimately decide what
can be posted under the name of Wikipedia and on its server, and its
board is only for a part elected by the participants in Wikipedia\footnote{\url{http://wikimediafoundation.org/wiki/Board_of_Trustees}}.
Quite surprisingly regarding their importance, especially in the open
source world, we can not find study of how Foundations manage open
online communities \citep[chapter 6, mentions this point however and gives a good start for Wikipedia.]{Reagle10}.
It is however clear that the leaders of these projects play an important
role in defining its goals and orientation (see \citealp{CrowstonHeckmanMisiolek10}
for an analysis of this aspect), especially in the case of Wikipedia,
where one of the two founders, Jimmy Wales, gave the vision \citep[chapter 1]{Reagle10}
and is still considered as the leader of the project, is the ultimate
decision maker \citep[chapter 6, which deals with Wikipedia leadership]{Reagle10}
and has a permanent seat in the board of trustees as founding member.

He is at the origin of the tables of law of the project, the ``five
pillars'' of Wikipedia\footnote{For the English: \url{http://en.Wikipedia.org/wiki/Wikipedia:Five_pillars},
but they are available in practically any language supported by the
project.}, defining the product (online encyclopedia) and its scope (neutral
point of view, no original research, accuracy, which are the three
core policy guiding the organization, according to \citealp{Reagle10}),
the producers and the users (anyone), and the process of production
(interaction and good faith), knowing that, as every project organization,
it has to be adaptable (no firm rules). \citet{CardonLevrel09,Cardon12}
propose a deep analysis of these explicit and implicit rules, showing
that these rules aim at involving any participant in the monitoring
and discussion of others' contribution, designing a procedural organization
(we will come back latter to this point). In other words, this organization
would be an attempt to create a ''supportive environment'' (\citealt{Reagle10b},
basing on \citealp{Gibb61}), i.e. an environment which privileges
''description (vs evaluation), problem orientation, spontaneity,
empathy, equality, provisionalism''. However, this also means that
the the foundations of the organization are constantly renegotiated
by the people (and by their behavior), leading ambiguity to be ''at
the heart of the policy process on Wikipedia'' \citep{MateiDobrescu11}.
It works because rules are mainly integrated by the persons in charge
(this is part of the process of involvement into the project), which
allows in the same time the maintaining of a common goal (the ideal
of consensus building and discussion) and the growing decentralization
of the day-to-day decisions due to the growing size of the project
\citep{ForteLarcoBruckman09}.

And when a very deep conflict appears, such as the case of the \textit{Jyllands-
Posten Muhammad Cartoon Controversy} \citep{MorganMasonNahon11} it
seems that the appeal to the values-in-practice, i.e. freedom of information
over multicultural inclusiveness (ibid, p. 7), thus to the common
rules, is a very powerful mean to gain the decision.

% Conclusion on that part, and history of the project.

The question is then to understand how this system is organize to
attract and retain these people, make them collaborate, and deal with
the growth of the population. As in many situations, the studies balance
between two positions: exploiting the data available, and the fact
that they are complete on the contribution, to provide general global
results on the participants, the products and the process, or deepening
the understanding beyond what is visible, thus trying to collect new
data, via exploratory methods, and compensating the loose in representativity
by a better understanding of the why or the how people do things.
Of course both are needed and complementary but, in general, we will
present the more global studies first, to have a global picture.