Regarding the evaluation of the articles, one can rely on the internal
evaluation proposed by Wikipedia projects (and especially on the Feature
Article system), on the criterion used to evaluate these articles.
However, \citet{Stviliaetal09} showed that the arguments of quality
to qualify an article as Feature vary from one language to another
(three languages studied, English, Arabic, and Korean). This work
was done comparing a subset of FA articles to non-FA articles (or
formally FA articles), so it needs to be extended to the whole dataset
and to other languages. \citealp[p. 5]{Hammwohner07}, showed that
for the English, French and Italian articles, ''the differences between
standard and featured articles are by far greater that those between
languages'', which suggest different models, each regrouping several
language projects, (and may be another field for culture studies approaches).
Finally, when external expertise is mobilized to evaluate the quality
of Feature Articles, as in \citeauthor{Lindsey10}'s work (2010),
strong variations appear: on a total of 22 usable responses collected
from a variety of discipline, only 12 of 22 were found to pass Wikipedia's
own featured article criteria, according to the author.

External criteria, mainly coming from library studies, are based on
\citet{Katz02a}'s criteria (e.g., purpose, authority, scope). But
\citet{WallaceVanFleet05,EhmannLargeBeheshti08} concluded that these
criteria are difficult to apply to Wikipedia, especially because there
is no authorship analysis possible, but also, as pointed by \citet{LewandowskiSpree11},
''due to the overall scale and the wide range of subject areas, most
of the studies focus on specialized fields of knowledge''. 

Considering this problem, two strategies have been developed to evaluate
the quality of Wikipedia, defining a subset of article to be analyzed,
either looking at randomly chosen articles, or looking at a sub-project
or a topic. In both cases, criteria have to be defined. The most comprehensive
attempt to do so may be the ones by \citet{Stviliaetal08}, and \citet{LewandowskiSpree11}.
The firsts proposed 11 criteria, based on more global analysis of
quality in on-line projects \citep{Stviliaetal07}, which compared
the FA articles to other articles regarding these criteria. The later
relied on these criteria, extended them to a list of 13 (see Table
\ref{tab:List-of-criteria-article-quality}, page \pageref{tab:List-of-criteria-article-quality}),
and evaluated the correlation between these criteria and the rank
in search engine, with a good correlation but a strong dispersion.
As stand by the authors, this does not solve the subjective aspect
of the criteria (or the fact their evaluation depends on the evaluator),
but the aim is to propose an evaluation grid.

\begin{table}
\caption{\label{tab:List-of-criteria-article-quality}List of applied quality
evaluation criteria for an article, from \citet[table 1, p. 10]{LewandowskiSpree11}.}

\begin{tabular}{|c|c|}
\hline 
\multirow{2}{*}{Labeling/lemmatization } & Obvious/non-ambiguous\tabularnewline
 & Common usage\tabularnewline
\hline 
\multirow{2}{*}{Scope Stays focused on the topic (W)} & Stays focused on the topic (W)\tabularnewline
 & No original research (W)\tabularnewline
\hline 
\multirow{2}{*}{Comprehensiveness} & Addresses the major aspects of the topic (W)\tabularnewline
 & Understandable as independent text\tabularnewline
\hline 
\multirow{3}{*}{Size} & Concise (W)\tabularnewline
 & No longer than 32 KB (W)\tabularnewline
 & Appropriate to the importance of topic (W)\tabularnewline
\hline 
\multirow{2}{*}{Accuracy} & Orthographically and grammatically correct (W)\tabularnewline
 & Consistency (concerning names, quotes, numbers, etc.) (W)\tabularnewline
\hline 
\multirow{2}{*}{Recency} & Up to date-ness of cited or recommended resources\tabularnewline
 & Up to date/developments of the last 3 month are covered\tabularnewline
\hline 
\multirow{6}{*}{Clarity and readability} & Concise head lead section (W)\tabularnewline
 & System of hierarchical headings (W)\tabularnewline
 & Informative headlines (W)\tabularnewline
 & Factual\tabularnewline
 & From the specific to the general\tabularnewline
 & Coherent writing\tabularnewline
\hline 
\multirow{9}{*}{Writing style} & News style/summary style (W)\tabularnewline
 & Formal, dispassionate, impersonal (W)\tabularnewline
 & Avoiding jargon\tabularnewline
 & Contextualization\tabularnewline
 & Concise\tabularnewline
 & Avoiding ambiguities\tabularnewline
 & Avoiding redundancies\tabularnewline
 & Descriptive, inspiring/interesting\tabularnewline
 & Clear/using examples\tabularnewline
\hline 
\multirow{2}{*}{Viewpoint and objectivity} & Neutral\tabularnewline
 & Fair and traceable presentation of controversial views\tabularnewline
\hline 
\multirow{4}{*}{Authority} & Verifiable facts (W)\tabularnewline
 & Reliable sources\tabularnewline
 & Informative academic writing style\tabularnewline
 & Longevity/stability\tabularnewline
\hline 
\multirow{4}{*}{Bibliographies} & Uniform way of citation (according to style guide)\tabularnewline
 & Quotations\tabularnewline
 & Further reading\tabularnewline
 & External links\tabularnewline
\hline 
\multirow{3}{*}{Access, organization, and accessibility} & Internal links\tabularnewline
 & External links\tabularnewline
 & Table of contents\tabularnewline
\hline 
\multirow{6}{*}{Additional material} & Pictures and graphics\tabularnewline
 & Self-explanatory images and graphics\tabularnewline
 & Captions (W)\tabularnewline
 & Copyright statement\tabularnewline
 & Special features\tabularnewline
 & Tabulary overviews\tabularnewline
\hline 
\end{tabular}\\
(Attributes marked with W are derived from Wikipedia)
\end{table}

\paragraph{Random analysis.}

The most famous of this type of evaluation is the one done by \citet{Giles05},
who sent articles from both Wikipedia and Encyclopaedia Britannica
to experts and compared their evaluation in terms of errors (factual
errors, critical omissions and misleading statements), but also structure
and writing, with no considerable difference between the two in terms
of errors, even if the Wikipedia articles were perceived as less well
written. Also using this random method, \citet{Chesney06} evaluated
that 13\% of the articles used in his study contained mistakes. Results
concerning the consistency and comprehensiveness of individual articles
have been generally regarded as satisfying \citep{Hammwohner07}. 

If \citet{Luytetal08}, who proposed a review of the literature on
Wikipedia accuracy, concluded that there ''is some evidence with
regard to formal accuracy (orthography) that Wikipedia is less reliable
than comparable works'', they joined \citet[p. 1668]{Fallis08} to
wonder if the assumption that a high number of orthographic mistakes
indicates an equally high number of factual mistakes is valid\footnote{This is, actually, a main difference with the measures of software
quality, such as the number of defects per thousand lines of code
\citep{DiazSligo97,Goranson97} or the probability of a fault in a
module \citep{Basili92,BasiliCalidieraRombach94}: in Wikipedia, technical
errors (syntax) do not mean, are not the sign of an error in the function,
the information delivered).}. However, some work is still needed to automate these syntax error
checks, in order to provide better and more global estimation of their
number (and also to allow the managers of the project to correct them
more efficiently).

\paragraph{Analyses by topic.}

The analyses by topics are usually micro-analyses and comparisons
with established encyclopedias, thus on articles or points where articles
exist in both publications. For instance, \citet{Bragues09} studied
seven entries of seven main philosophers and compared them with academic
references, evaluating the coverage of these references content by
Wikipedia above 50\%, and without critical errors.

Regarding history and accuracy, Wikipedia could be ranked, in 2008,
between Encarta and American National Biography Online for the USA's
history, even if poorly written \citep{Rosenzweig06}, which was also
the conclusion of \citet{Rector08}, even if a strong set of references
to scientific publications is lacking, maybe because these are not
freely available \citep{LuytTan10}. It seems, however, to remain
close to dominant version of the history (what exist in the traditional
publications), according to \citet{Luyt11}\footnote{This is the explanation he gave to the fact that the official version
of the history is the only presented for Singapour, when other perspectives
are taking into account for Philippine: for the latter, these various
points of view are already well existing in the tradition publications.}. Conclusions are rather the same on the brain and behavioral sciences
\citep[2010b]{StankusSpiegel10a}\nocite{StankusSpiegel10b} where,
if Wikipedia relies more strongly on scientific journals in that field,
it has less citation that a written-by-experts encyclopedia (Scholarpedia). 

One area where accuracy is a main concern is medical information,
where Wikipedia is today one of the main source \citep{LaurentVickers09}.
A conclusion of its quality (in drug information), by \citet{Clausonetal08},
is the following: ''Wikipedia has a more narrow scope, is less complete,
and has more errors of omission than the comparator database. Wikipedia
may be a useful point of engagement for consumers, but is not authoritative
and should only be a supplemental source of drug information.'' More
precisely, if looking at the review of the medical literature on Wikipedia
\citet{KupferbergProtus11} proposed, it seems that this site does
not contain more error than other online Web site, but lacks depth
(same result found by \citet{Leithneretal10} on the more specific
topic of osteosarcoma). However, this does not mean that the information
is complete. \citet{Lavsaetal11}, studying the information available
in the English Wikipedia on the 20 most common drugs found that the
encyclopedia ''does not provide consistently accurate, complete,
and referenced medication information'', especially on drug interaction,
contraindications or dosing and warned against its use by pharmacy
students. \citet{Devganetal07} found Wikipedia ''accurate though
often incomplete medical reference, with a remarkably high level of
internal validity'' regarding surgical references.