\subsection{Generating and evaluating the initial set of contributors profiles}

As with the first community manager, we begin by generating an initial set of $25$ contributor profiles, which \textsc{CM2} then assigns to one the three selected categories, as illustrated in Table~\ref{tab:ex2-step1}. In order to better compare the first two community managers, we have used exactly the same initial set of profiles for both community managers.

\begin{table}
\caption{The initial set of contributor profiles and their assignment by the community manager;}\label{tab:ex2-step1}
\small

\begin{longtable}{cccccc|c|c|c}
Profile& \multicolumn{5}{c}{Criteria} & \multicolumn{3}{c}{Category}\\
number& $c_1$ & $c_2$ & $c_3$ & $c_4$ & $c_5$ & \multicolumn{1}{c}{Good} & \multicolumn{1}{c}{Neutral} & \multicolumn{1}{c}{Bad}\\\hline
1 & v. good &    good &  v. bad & v. good & neutral & \correct & & \\\hline
2 &    bad & v. good & neutral &  v. bad & neutral &  &  & \correct \\\hline
3 &    bad &     bad &     bad &     bad &    good &  &  & \correct \\\hline
4 &    bad &     bad &  v. bad & v. good & neutral &  &  & \correct \\\hline
5 &   good &  v. bad & v. good &     bad &     bad &  & \correct &  \\\hline
6 &v. good &    good & v. good & neutral & v. good & \correct &  &  \\\hline
7 &   good & neutral &     bad & neutral & v. good & \correct &  &  \\\hline
8 &neutral & neutral &    good &     bad &    good &  & \correct &  \\\hline
9 &neutral & v. good & neutral &    good &     bad & \correct &  &  \\\hline
10& v. bad &    good & v. good &  v. bad &     bad &  &  & \correct \\\hline
11&   good &    good &    good &  v. bad & v. good & \correct &  &  \\\hline
12&neutral & v. good &    good &  v. bad &    good & \correct &  &  \\\hline
13&   good &    good & neutral & neutral &  v. bad & \correct &  &  \\\hline
14&v. good & neutral &  v. bad &     bad &    good & \correct &  &  \\\hline
15&neutral &     bad &     bad &  v. bad & neutral &  &  & \correct \\\hline
16&    bad & v. good &    good & v. good &  v. bad &  & \correct &  \\\hline
17&neutral &     bad & neutral &    good & neutral &  & \correct &  \\\hline
18&v. good &  v. bad & v. good &    good &     bad &  & \correct &  \\\hline
19&v. good &  v. bad & neutral & neutral &  v. bad &  & \correct &  \\\hline
20& v. bad & v. good & v. good &     bad & v. good &  &  & \correct \\\hline
21& v. bad & neutral &  v. bad & neutral &  v. bad &  &  & \correct \\\hline
22& v. bad &     bad &  v. bad & v. good &    good &  &  & \correct \\\hline
23& v. bad &  v. bad &    good &    good & v. good &  &  & \correct \\\hline
24&   good &  v. bad &     bad &    good &  v. bad &  & \correct &  \\\hline
25&    bad & neutral &     bad & v. good &     bad &  &  & \correct \\\hline
\end{longtable}
\end{table}

\subsection{Determining the complexity of the first model}

We continued by testing whether an MR-Sort model was able to represent the provided assignments and found that only at most $24$ out of the $25$ were captured. In order to render an MR-Sort model applicable to all the $25$ profiles, only one out of two profiles would need to be assigned differently by the community manager, as illustrated in Table~\ref{tab:ex2-step2}.

\begin{table}
\caption{2}\label{tab:ex2-step2}
\small

\begin{tabular}{cccccc|cc}
Profile& \multicolumn{5}{c}{Criteria} & \multicolumn{2}{|c}{Category} \\
number& $c_1$ & $c_2$ & $c_3$ & $c_4$ & $c_5$ & \multicolumn{1}{c}{original} & \multicolumn{1}{c}{alternative} \\\hline
8 &neutral & neutral &    good &     bad &    good & Neutral & Good \\\hline
14 &v. good & neutral &  v. bad &     bad &    good & Good & Neutral \\        
\end{tabular}
\end{table}

The community manager had already expressed a hesitation in assigning the eighth profile, therefore, upon presenting these two profiles to him, he agreed to change the assignment of the first profile.

The resulting MR-Sort model is presented in Fig.~\ref{fig:ex2-model1}.

\begin{figure}
\centering

\begin{minipage}[c]{0.4\columnwidth}
\vspace{0pt}
\begin{tikzpicture}
\begin{axis}[title ={Bad-Neutral}, height=\textwidth,width=\textwidth, xmin = 0.5, xmax = 5.5, ymin = -2.5, ymax = 2.5, every axis x label/.style={at={(ticklabel* cs:0.97)},anchor=south},xtick={1,2,3,4,5}, xticklabels={$c_1$,$c_2$,$c_3$,$c_4$,$c_5$}, ytick={-2,-1,0,1,2}, xmajorgrids = true, axis line style = { draw = none }, yticklabels = {vb, b, n, g, vg}, ymajorgrids = true]
\addplot[name path=T]
coordinates {
	(1,3)
	(2,3)
	(3,3)
	(4,3)
	(5,3)
};
\addplot[name path=B]
coordinates {
	(1,-3)
	(2,-3)
	(3,-3)
	(4,-3)
	(5,-3)
};
\addplot[name path=P, black, solid, mark = *]
coordinates {
	(1,-0.5)
	(2,-0.5)
	(3,-0.5)
	(4,-0.5)
	(5,-2)
};
\end{axis}
\end{tikzpicture}
\end{minipage}
\begin{minipage}[c]{0.4\columnwidth}
\vspace{0pt}

\begin{tikzpicture}
\begin{axis}[title ={Neutral-Good}, height=\textwidth,width=\textwidth, xmin = 0.5, xmax = 5.5, ymin = -2.5, ymax = 2.5, every axis x label/.style={at={(ticklabel* cs:0.97)},anchor=south},xtick={1,2,3,4,5}, xticklabels={$c_1$,$c_2$,$c_3$,$c_4$,$c_5$}, ytick={-2,-1,0,1,2}, xmajorgrids = true, axis line style = { draw = none }, yticklabels = {vb, b, n, g, vg}, ymajorgrids = true]
\addplot[name path=T]
coordinates {
	(1,3)
	(2,3)
	(3,3)
	(4,3)
	(5,3)
};
\addplot[name path=B]
coordinates {
	(1,-3)
	(2,-3)
	(3,-3)
	(4,-3)
	(5,-3)
};
\addplot[name path=P, black, solid, mark = *]
coordinates {
	(1,-0.5)
	(2,-0.5)
	(3,-0.5)
	(4,0.5)
	(5,0.5)
};
\end{axis}
\end{tikzpicture}
\end{minipage}
\begin{minipage}[c]{0.07\columnwidth}
\vspace{0pt}

\begin{tabular}{c|c}
$\lambda$ & 0.6 \\\hline
$c_1$ & 0.40 \\
$c_2$ & 0.15 \\
$c_3$ & 0.15 \\
$c_4$ & 0.15 \\
$c_5$ & 0.15 \\
\end{tabular}
\end{minipage}
\caption{First preference model of \DB (MR-Sort).}\label{fig:ex2-model1}
\end{figure}

We decide to continue with a new iteration.

\subsection{Generating and evaluating an additional set of profiles}

An additional set of $10$ profiles is generated, based on the previously created model. This set is presented to the CM who then assigns them as seen in Table~\ref{tab:ex2-step3}.

\begin{table}
\caption{3}\label{tab:ex2-step3}
\small

\begin{tabular}{cccccc|c|c|c}
Profile& \multicolumn{5}{c}{Criteria} & \multicolumn{3}{c}{Category}\\
number& $c_1$ & $c_2$ & $c_3$ & $c_4$ & $c_5$ & \multicolumn{1}{c}{Good} & \multicolumn{1}{c}{Neutral} & \multicolumn{1}{c}{Bad}\\\hline
26& bad & bad & bad & v. good & v. good & & & \correct \\\hline
27& v. bad & neutral & good & neutral & good & & & \correct \\\hline
28& v. good & bad & neutral & bad & neutral & & \correct & \\\hline
29& neutral & neutral & v. bad & v. bad & good & & \correct & \\\hline
30& bad & bad & v. good & bad & v. good & & & \correct \\\hline
31& neutral & neutral & good & v. bad & v. bad & & \correct & \\\hline
32& neutral & v. bad & neutral & v. bad & v. bad & & & \correct \\\hline
33& v. bad & neutral & neutral & neutral & v. bad & & & \correct \\\hline
34& neutral & neutral & v. bad & neutral & v. bad & & & \correct \\\hline
35& neutral & v. bad & v. bad & neutral & good & & \correct & \\\hline
\end{tabular}
\end{table}

\subsection{Determining the complexity of the second model}

We combine the initial set of $25$ profiles with the new set of $10$ and test whether an MR-Sort model is still able to capture them. Only $33$ out of the $35$ profiles can be represented by this type of model, hence we proceed to determining whether the community manager would accept to adapt a minimal number of his assignments so that such a model could be used further. Only a set of two profiles would make this possible (Table~\ref{tab:ex2-step4a}).

\begin{table}
\caption{4}\label{tab:ex2-step4a}
\small

\begin{tabular}{ccccccc|cc}
&Profile& \multicolumn{5}{c}{Criteria} & \multicolumn{2}{|c}{Category} \\
&number& $c_1$ & $c_2$ & $c_3$ & $c_4$ & $c_5$ & \multicolumn{1}{c}{original} & \multicolumn{1}{c}{alternative} \\\hline
\multirow{2}{*}{First set}& 16&    bad & v. good &    good & v. good &  v. bad &  Neutral & Bad \\
& 34& neutral & neutral & v. bad & neutral & v. bad &  Bad & Good \\
\end{tabular}
\end{table}

The community manager expressed a willingness to change the assignment of the first profile from this set, if needed, however, the radical change in assignment of the second profiles was too drastic. We therefore proceed to determining whether an MR-Sort model with vetoes, or one with dictators, could be used at this point. Both of these models also appeared to require changes to the assignments of the community managers, however, as the model with dictators needed two profiles to be changed at once, whereas the model with vetoes only needed one, we focused initially on the latter model. Changing the assignment of either of three profiles (Table~\ref{tab:ex2-step4b} would render an MR-Sort model with vetoes applicable in our situation.

\begin{table}
\caption{4}\label{tab:ex2-step4b}
\small

\begin{tabular}{cccccc|cc}
Profile& \multicolumn{5}{c}{Criteria} & \multicolumn{2}{|c}{Category} \\
number& $c_1$ & $c_2$ & $c_3$ & $c_4$ & $c_5$ & \multicolumn{1}{c}{original} & \multicolumn{1}{c}{alternative} \\\hline
 16&    bad & v. good &    good & v. good &  v. bad &  Neutral & Bad \\\hline
 32& neutral & v. bad & neutral & v. bad & v. bad &  Bad & Neutral \\\hline
 34& neutral & neutral & v. bad & neutral & v. bad &  Bad & Neutral \\
\end{tabular}
\end{table}

As the community manager previously expressed a hesitation on the assignment of profiles $16$, he quickly agrees to the proposed change. The model that reflects these assignments is depicted in Fig.~\ref{fig:ex2-model2}.

\begin{figure}
\centering

\begin{minipage}[c]{0.4\columnwidth}
\vspace{0pt}
\begin{tikzpicture}
\begin{axis}[title ={Bad-Neutral}, height=\textwidth,width=\textwidth, xmin = 0.5, xmax = 5.5, ymin = -2.5, ymax = 2.5, every axis x label/.style={at={(ticklabel* cs:0.97)},anchor=south},xtick={1,2,3,4,5}, xticklabels={$c_1$,$c_2$,$c_3$,$c_4$,$c_5$}, ytick={-2,-1,0,1,2}, xmajorgrids = true, axis line style = { draw = none }, yticklabels = {vb, b, n, g, vg}, ymajorgrids = true]
\addplot[name path=T]
coordinates {
	(1,3)
	(2,3)
	(3,3)
	(4,3)
	(5,3)
};
\addplot[name path=B]
coordinates {
	(1,-3)
	(2,-3)
	(3,-3)
	(4,-3)
	(5,-3)
};
\addplot[name path=P, black, solid, mark = *]
coordinates {
	(1,0.5)
	(2,-1.5)
	(3,-0.5)
	(4,-2)
	(5,0.5)
};
\addplot[name path=V]
coordinates {
	(0,-1)
	(1,-3)
	(2,-3)
	(3,-3)
	(4,-3)
	(5,-3)
};
\addplot[black] fill between[of=V and B];
\end{axis}
\end{tikzpicture}
\end{minipage}
\begin{minipage}[c]{0.4\columnwidth}
\vspace{0pt}

\begin{tikzpicture}
\begin{axis}[title ={Neutral-Good}, height=\textwidth,width=\textwidth, xmin = 0.5, xmax = 5.5, ymin = -2.5, ymax = 2.5, every axis x label/.style={at={(ticklabel* cs:0.97)},anchor=south},xtick={1,2,3,4,5}, xticklabels={$c_1$,$c_2$,$c_3$,$c_4$,$c_5$}, ytick={-2,-1,0,1,2}, xmajorgrids = true, axis line style = { draw = none }, yticklabels = {vb, b, n, g, vg}, ymajorgrids = true]
\addplot[name path=T]
coordinates {
	(1,3)
	(2,3)
	(3,3)
	(4,3)
	(5,3)
};
\addplot[name path=B]
coordinates {
	(1,-3)
	(2,-3)
	(3,-3)
	(4,-3)
	(5,-3)
};
\addplot[name path=P, black, solid, mark = *]
coordinates {
	(1,0.5)
	(2,-0.5)
	(3,-0.5)
	(4,0.5)
	(5,0.5)
};
\addplot[name path=V]
coordinates {
	(1,-1)
	(2,-2)
	(3,-3)
	(4,-3)
	(5,-3)
};
\addplot[black] fill between[of=V and B];
\end{axis}
\end{tikzpicture}
\end{minipage}
\begin{minipage}[c]{0.07\columnwidth}
\vspace{0pt}

\begin{tabular}{c|c}
$\lambda$ & 0.5 \\\hline
$c_1$ & 0.16733 \\
$c_2$ & 0.16733 \\
$c_3$ & 0.16733 \\
$c_4$ & 0.33267 \\
$c_5$ & 0.16733
\end{tabular}
\end{minipage}

\caption{Second preference model of \DB (MR-Sort with vetoes).}\label{fig:ex2-model2}
\end{figure}

The generated model appears to further illustrate the existence of vetoes on the first two criteria, something which came out of the reasoning the community manager presented when making some of his assignments. The community manager agrees to continue with several more steps of this protocol.

\subsection{Generating and evaluating a second complementary set of profiles}

We generate an additional set of $10$ profiles, which the community manager assigns as illustrated in Table~\ref{tab:ex2-step5}.

\begin{table}
\caption{5}\label{tab:ex2-step5}
\small

\begin{tabular}{cccccc|c|c|c}
Profile& \multicolumn{5}{c}{Criteria} & \multicolumn{3}{c}{Category}\\
number& $c_1$ & $c_2$ & $c_3$ & $c_4$ & $c_5$ & \multicolumn{1}{c}{Good} & \multicolumn{1}{c}{Neutral} & \multicolumn{1}{c}{Bad}\\\hline
36 & neutral & v. bad & v. bad & v. bad & good &  & \correct &  \\\hline
37 & neutral & good & neutral & v. bad & good &  & \correct &  \\\hline
38 & v. good & v. bad & v. good & v. good & v. good & \correct &  &  \\\hline
39 & bad & v. good & v. good & v. good & v. good &  & \correct &  \\\hline
40 & neutral & neutral & bad & v. good & v. good &  & \correct &  \\\hline
41 & good & good & neutral & v. bad & v. bad & \correct &  &  \\\hline
42 & neutral & v. bad & bad & v. good & v. good &  & \correct &  \\\hline
43 & good & bad & v. bad & v. bad & v. bad &  & \correct &  \\\hline
44 & neutral & bad & neutral & v. bad & v. bad &  & \correct &  \\\hline
45 & v. good & v. bad & bad & v. good & neutral & \correct &  &  \\\hline
\end{tabular}
\end{table}

\subsection{Determining the complexity of the third model}

After adding the $10$ new profiles and their assignments to the existing ones, we find that an MR-Sort model with vetoes is not able to represent all of these assignments and that we would require sets of at least three alternatives to have their assignments altered in order for such a model to be used. These set are presented in Table~\ref{tab:ex2-step6}.

\begin{table}
\caption{6}\label{tab:ex-phase1-step6}
\small

\begin{tabular}{ccccccc|cc}
&Profile& \multicolumn{5}{c}{Criteria} & \multicolumn{2}{|c}{Category} \\
&number& $c_1$ & $c_2$ & $c_3$ & $c_4$ & $c_5$ & \multicolumn{1}{c}{original} & \multicolumn{1}{c}{alternative} \\\hline
\multirow{1}{*}{First set}&9 &neutral & v. good & neutral &    good &     bad &  Good & Neutral \\
&32& neutral & v. bad & neutral & v. bad & v. bad & Bad & Neutral \\
&39 & bad & v. good & v. good & v. good & v. good & Neutral & Bad \\\hline

\multirow{1}{*}{Second set}&9 &neutral & v. good & neutral &    good &     bad &  Good & Neutral \\
&12&neutral & v. good &    good &  v. bad &    good & Good & Neutral \\
&39 & bad & v. good & v. good & v. good & v. good & Neutral & Bad \\\hline
         
\multirow{1}{*}{Third set}&38 & v. good & v. bad & v. good & v. good & v. good &  Good & Neutral \\
&39 & bad & v. good & v. good & v. good & v. good & Neutral & Bad \\
&45 & v. good & v. bad & bad & v. good & neutral & Good & Neutral \\\hline
\end{tabular}
\end{table}

As profile $39$ appears in all three sets, we first ask the community manager if he hesitated in assigning this profile or not. The answer is negative, hence we check whether a more complex model, which weakens the effect of the veto so that the model becomes more flexible, may be used. An MR-Sort model with veto weakened by a dictator is able to fully reflect all of the assignments of the community manager. We illustrate this model in Fig.~\ref{fig:ex2-model3}.

\begin{figure}
\centering

\begin{minipage}[c]{0.4\columnwidth}
\vspace{0pt}
\begin{tikzpicture}
\begin{axis}[title ={Bad-Neutral}, height=\textwidth,width=\textwidth, xmin = 0.5, xmax = 5.5, ymin = -2.5, ymax = 2.5, every axis x label/.style={at={(ticklabel* cs:0.97)},anchor=south},xtick={1,2,3,4,5}, xticklabels={$c_1$,$c_2$,$c_3$,$c_4$,$c_5$}, ytick={-2,-1,0,1,2}, xmajorgrids = true, axis line style = { draw = none }, yticklabels = {vb, b, n, g, vg}, ymajorgrids = true]
\addplot[name path=T]
coordinates {
	(1,3)
	(2,3)
	(3,3)
	(4,3)
	(5,3)
};
\addplot[name path=B]
coordinates {
	(1,-3)
	(2,-3)
	(3,-3)
	(4,-3)
	(5,-3)
};
\addplot[name path=P, black, solid, mark = *]
coordinates {
	(1,-0.5)
	(2,-1.5)
	(3,1.5)
	(4,-0.5)
	(5,-0.5)
};
\addplot[name path=V]
coordinates {
	(1,-1)
	(2,-3)
	(3,-1)
	(4,-3)
	(5,-3)
};
\addplot[name path=D]
coordinates {
	(1,1)
	(2,2)
	(3,3)
	(4,3)
	(5,1)
};
\addplot[black] fill between[of=V and B];
\addplot[pattern = north east lines] fill between[of=D and T];
\end{axis}
\end{tikzpicture}
\end{minipage}
\begin{minipage}[c]{0.4\columnwidth}
\vspace{0pt}

\begin{tikzpicture}
\begin{axis}[title ={Neutral-Good}, height=\textwidth,width=\textwidth, xmin = 0.5, xmax = 5.5, ymin = -2.5, ymax = 2.5, every axis x label/.style={at={(ticklabel* cs:0.97)},anchor=south},xtick={1,2,3,4,5}, xticklabels={$c_1$,$c_2$,$c_3$,$c_4$,$c_5$}, ytick={-2,-1,0,1,2}, xmajorgrids = true, axis line style = { draw = none }, yticklabels = {vb, b, n, g, vg}, ymajorgrids = true]
\addplot[name path=T]
coordinates {
	(1,3)
	(2,3)
	(3,3)
	(4,3)
	(5,3)
};
\addplot[name path=B]
coordinates {
	(1,-3)
	(2,-3)
	(3,-3)
	(4,-3)
	(5,-3)
};
\addplot[name path=P, black, solid, mark = *]
coordinates {
	(1,-0.5)
	(2,0.5)
	(3,2.5)
	(4,1.5)
	(5,0.5)
};
\addplot[name path=V]
coordinates {
	(1,-1)
	(2,-3)
	(3,-1)
	(4,-2)
	(5,-3)
};
\addplot[name path=D]
coordinates {
	(1,1)
	(2,2)
	(3,3)
	(4,3)
	(5,3)
};
\addplot[black] fill between[of=V and B];
\addplot[pattern = north east lines] fill between[of=D and T];
\end{axis}
\end{tikzpicture}
\end{minipage}
\begin{minipage}[c]{0.07\columnwidth}
\vspace{0pt}

\begin{tabular}{c|c}
$\lambda$ & 0.55 \\\hline
$c_1$ & 0.40 \\
$c_2$ & 0.15 \\
$c_3$ & 0.15 \\
$c_4$ & 0.15 \\
$c_5$ & 0.15
\end{tabular}
\end{minipage}

\caption{Third preference model of \DB (MR-Sort with vetoes weakened by dictators).}\label{fig:ex2-model3}
\end{figure}

At this point, we may additionally construct a total of $27$ new profiles in order to be able to fix the model parameters. The community manager, however, express an interest in seeing the currently generated model, as it should already reflect a large portion of his perspective on this problem.

\subsection{Validating the final model}

We present the model from Fig.~\ref{fig:ex2-model3} as a series of rules, illustrated graphically in Fig.~\ref{fig:ex2-rules1}.

\begin{figure}
	\centering
	
		\hrule
		\vspace{1ex}
	
	{\bf Good Contributors}
	
		\vspace{1ex}
		\hrule
		\vspace{1ex}

\noindent\begin{tikzpicture}
\begin{axis}[width = 4.5cm, ytick = {-2,-1,0,1,2}, xtick = {1,2,3,4,5}, xmin = 0.5, xmax = 5.500000, ymin = -2.5, ymax = 2.5, axis line style = {draw = none}, xmajorgrids, ymajorgrids, separate axis lines, yticklabels = {vb, b, n, g, vg}, xticklabels = {$c_1$,$c_2$,$c_3$,$c_4$,$c_5$}]
\addplot[name path = A, black, no markers, solid, very thick]
coordinates {
	(1,2)
	(2,2)
	(3,2)
	(4,2)
	(5,2)
};
\addplot[name path = B, black, no markers, solid, very thick]
coordinates {
	(1,0)
	(2,-2)
	(3,0)
	(4,0)
	(5,1)
};
\addplot[name path = C, black, mark = *, only marks, very thick]
coordinates {
	(1,0)
	(1,1)
	(1,2)
	(2,-2)
	(2,-1)
	(2,0)
	(2,1)
	(2,2)
	(3,0)
	(3,1)
	(3,2)
	(4,0)
	(4,1)
	(4,2)
	(5,1)
	(5,2)
};
\addplot[white!70!black] fill between[of=A and B];
\end{axis}
\end{tikzpicture}
\noindent\begin{tikzpicture}
\begin{axis}[width = 4.5cm, ytick = {-2,-1,0,1,2}, xtick = {1,2,3,4,5}, xmin = 0.5, xmax = 5.500000, ymin = -2.5, ymax = 2.5, axis line style = {draw = none}, xmajorgrids, ymajorgrids, separate axis lines, yticklabels = {vb, b, n, g, vg}, xticklabels = {$c_1$,$c_2$,$c_3$,$c_4$,$c_5$}]
\addplot[name path = A, black, no markers, solid, very thick]
coordinates {
	(1,2)
	(2,2)
	(3,2)
	(4,2)
	(5,2)
};
\addplot[name path = B, black, no markers, solid, very thick]
coordinates {
	(1,0)
	(2,-2)
	(3,0)
	(4,2)
	(5,-2)
};
\addplot[name path = C, black, mark = *, only marks, very thick]
coordinates {
	(1,0)
	(1,1)
	(1,2)
	(2,-2)
	(2,-1)
	(2,0)
	(2,1)
	(2,2)
	(3,0)
	(3,1)
	(3,2)
	(4,2)
	(5,-2)
	(5,-1)
	(5,0)
	(5,1)
	(5,2)
};
\addplot[white!70!black] fill between[of=A and B];
\end{axis}
\end{tikzpicture}
\noindent\begin{tikzpicture}
\begin{axis}[width = 4.5cm, ytick = {-2,-1,0,1,2}, xtick = {1,2,3,4,5}, xmin = 0.5, xmax = 5.500000, ymin = -2.5, ymax = 2.5, axis line style = {draw = none}, xmajorgrids, ymajorgrids, separate axis lines, yticklabels = {vb, b, n, g, vg}, xticklabels = {$c_1$,$c_2$,$c_3$,$c_4$,$c_5$}]
\addplot[name path = A, black, no markers, solid, very thick]
coordinates {
	(1,2)
	(2,2)
	(3,2)
	(4,2)
	(5,2)
};
\addplot[name path = B, black, no markers, solid, very thick]
coordinates {
	(1,0)
	(2,1)
	(3,0)
	(4,0)
	(5,-2)
};
\addplot[name path = C, black, mark = *, only marks, very thick]
coordinates {
	(1,0)
	(1,1)
	(1,2)
	(2,1)
	(2,2)
	(3,0)
	(3,1)
	(3,2)
	(4,0)
	(4,1)
	(4,2)
	(5,-2)
	(5,-1)
	(5,0)
	(5,1)
	(5,2)
};
\addplot[white!70!black] fill between[of=A and B];
\end{axis}
\end{tikzpicture}
\noindent\begin{tikzpicture}
\begin{axis}[width = 4.5cm, ytick = {-2,-1,0,1,2}, xtick = {1,2,3,4,5}, xmin = 0.5, xmax = 5.500000, ymin = -2.5, ymax = 2.5, axis line style = {draw = none}, xmajorgrids, ymajorgrids, separate axis lines, yticklabels = {vb, b, n, g, vg}, xticklabels = {$c_1$,$c_2$,$c_3$,$c_4$,$c_5$}]
\addplot[name path = A, black, no markers, solid, very thick]
coordinates {
	(1,2)
	(2,2)
	(3,2)
	(4,2)
	(5,2)
};
\addplot[name path = B, black, no markers, solid, very thick]
coordinates {
	(1,1)
	(2,-2)
	(3,-2)
	(4,-2)
	(5,1)
};
\addplot[name path = C, black, mark = *, only marks, very thick]
coordinates {
	(1,1)
	(1,2)
	(2,-2)
	(2,-1)
	(2,0)
	(2,1)
	(2,2)
	(3,-2)
	(3,-1)
	(3,0)
	(3,1)
	(3,2)
	(4,-2)
	(4,-1)
	(4,0)
	(4,1)
	(4,2)
	(5,1)
	(5,2)
};
\addplot[white!70!black] fill between[of=A and B];
\end{axis}
\end{tikzpicture}

\noindent\begin{tikzpicture}
\begin{axis}[width = 4.5cm, ytick = {-2,-1,0,1,2}, xtick = {1,2,3,4,5}, xmin = 0.5, xmax = 5.500000, ymin = -2.5, ymax = 2.5, axis line style = {draw = none}, xmajorgrids, ymajorgrids, separate axis lines, yticklabels = {vb, b, n, g, vg}, xticklabels = {$c_1$,$c_2$,$c_3$,$c_4$,$c_5$}]
\addplot[name path = A, black, no markers, solid, very thick]
coordinates {
	(1,2)
	(2,2)
	(3,2)
	(4,2)
	(5,2)
};
\addplot[name path = B, black, no markers, solid, very thick]
coordinates {
	(1,0)
	(2,2)
	(3,-2)
	(4,-2)
	(5,-2)
};
\addplot[name path = C, black, mark = *, only marks, very thick]
coordinates {
	(1,0)
	(1,1)
	(1,2)
	(2,2)
	(3,-2)
	(3,-1)
	(3,0)
	(3,1)
	(3,2)
	(4,-2)
	(4,-1)
	(4,0)
	(4,1)
	(4,2)
	(5,-2)
	(5,-1)
	(5,0)
	(5,1)
	(5,2)
};
\addplot[white!70!black] fill between[of=A and B];
\end{axis}
\end{tikzpicture}
	
		\vspace{1ex}
		\hrule
		\vspace{1ex}
	
	{\bf Neutral Contributors}
	
		\vspace{1ex}
		\hrule
		\vspace{1ex}
	
\noindent\begin{tikzpicture}
\begin{axis}[width = 4.5cm, ytick = {-2,-1,0,1,2}, xtick = {1,2,3,4,5}, xmin = 0.5, xmax = 5.500000, ymin = -2.5, ymax = 2.5, axis line style = {draw = none}, xmajorgrids, ymajorgrids, separate axis lines, yticklabels = {vb, b, n, g, vg}, xticklabels = {$c_1$,$c_2$,$c_3$,$c_4$,$c_5$}]
\addplot[name path = A, black, no markers, solid, very thick]
coordinates {
	(1,2)
	(2,0)
	(3,2)
	(4,1)
	(5,0)
};
\addplot[name path = B, black, no markers, solid, very thick]
coordinates {
	(1,0)
	(2,-2)
	(3,0)
	(4,0)
	(5,-2)
};
\addplot[name path = C, black, mark = *, only marks, very thick]
coordinates {
	(1,0)
	(1,1)
	(1,2)
	(2,-2)
	(2,-1)
	(2,0)
	(3,0)
	(3,1)
	(3,2)
	(4,0)
	(4,1)
	(5,-2)
	(5,-1)
	(5,0)
};
\addplot[white!70!black] fill between[of=A and B];
\end{axis}
\end{tikzpicture}
\noindent\begin{tikzpicture}
\begin{axis}[width = 4.5cm, ytick = {-2,-1,0,1,2}, xtick = {1,2,3,4,5}, xmin = 0.5, xmax = 5.500000, ymin = -2.5, ymax = 2.5, axis line style = {draw = none}, xmajorgrids, ymajorgrids, separate axis lines, yticklabels = {vb, b, n, g, vg}, xticklabels = {$c_1$,$c_2$,$c_3$,$c_4$,$c_5$}]
\addplot[name path = A, black, no markers, solid, very thick]
coordinates {
	(1,0)
	(2,1)
	(3,2)
	(4,-1)
	(5,2)
};
\addplot[name path = B, black, no markers, solid, very thick]
coordinates {
	(1,0)
	(2,-2)
	(3,0)
	(4,-2)
	(5,0)
};
\addplot[name path = C, black, mark = *, only marks, very thick]
coordinates {
	(1,0)
	(2,-2)
	(2,-1)
	(2,0)
	(2,1)
	(3,0)
	(3,1)
	(3,2)
	(4,-2)
	(4,-1)
	(5,0)
	(5,1)
	(5,2)
};
\addplot[white!70!black] fill between[of=A and B];
\end{axis}
\end{tikzpicture}
\noindent\begin{tikzpicture}
\begin{axis}[width = 4.5cm, ytick = {-2,-1,0,1,2}, xtick = {1,2,3,4,5}, xmin = 0.5, xmax = 5.500000, ymin = -2.5, ymax = 2.5, axis line style = {draw = none}, xmajorgrids, ymajorgrids, separate axis lines, yticklabels = {vb, b, n, g, vg}, xticklabels = {$c_1$,$c_2$,$c_3$,$c_4$,$c_5$}]
\addplot[name path = A, black, no markers, solid, very thick]
coordinates {
	(1,0)
	(2,1)
	(3,2)
	(4,-1)
	(5,2)
};
\addplot[name path = B, black, no markers, solid, very thick]
coordinates {
	(1,0)
	(2,-2)
	(3,2)
	(4,-2)
	(5,-2)
};
\addplot[name path = C, black, mark = *, only marks, very thick]
coordinates {
	(1,0)
	(2,-2)
	(2,-1)
	(2,0)
	(2,1)
	(3,2)
	(4,-2)
	(4,-1)
	(5,-2)
	(5,-1)
	(5,0)
	(5,1)
	(5,2)
};
\addplot[white!70!black] fill between[of=A and B];
\end{axis}
\end{tikzpicture}
\noindent\begin{tikzpicture}
\begin{axis}[width = 4.5cm, ytick = {-2,-1,0,1,2}, xtick = {1,2,3,4,5}, xmin = 0.5, xmax = 5.500000, ymin = -2.5, ymax = 2.5, axis line style = {draw = none}, xmajorgrids, ymajorgrids, separate axis lines, yticklabels = {vb, b, n, g, vg}, xticklabels = {$c_1$,$c_2$,$c_3$,$c_4$,$c_5$}]
\addplot[name path = A, black, no markers, solid, very thick]
coordinates {
	(1,0)
	(2,1)
	(3,2)
	(4,-1)
	(5,2)
};
\addplot[name path = B, black, no markers, solid, very thick]
coordinates {
	(1,0)
	(2,-1)
	(3,0)
	(4,-2)
	(5,-2)
};
\addplot[name path = C, black, mark = *, only marks, very thick]
coordinates {
	(1,0)
	(2,-1)
	(2,0)
	(2,1)
	(3,0)
	(3,1)
	(3,2)
	(4,-2)
	(4,-1)
	(5,-2)
	(5,-1)
	(5,0)
	(5,1)
	(5,2)
};
\addplot[white!70!black] fill between[of=A and B];
\end{axis}
\end{tikzpicture}

\noindent\begin{tikzpicture}
\begin{axis}[width = 4.5cm, ytick = {-2,-1,0,1,2}, xtick = {1,2,3,4,5}, xmin = 0.5, xmax = 5.500000, ymin = -2.5, ymax = 2.5, axis line style = {draw = none}, xmajorgrids, ymajorgrids, separate axis lines, yticklabels = {vb, b, n, g, vg}, xticklabels = {$c_1$,$c_2$,$c_3$,$c_4$,$c_5$}]
\addplot[name path = A, black, no markers, solid, very thick]
coordinates {
	(1,2)
	(2,0)
	(3,2)
	(4,-1)
	(5,0)
};
\addplot[name path = B, black, no markers, solid, very thick]
coordinates {
	(1,1)
	(2,-2)
	(3,0)
	(4,-2)
	(5,0)
};
\addplot[name path = C, black, mark = *, only marks, very thick]
coordinates {
	(1,1)
	(1,2)
	(2,-2)
	(2,-1)
	(2,0)
	(3,0)
	(3,1)
	(3,2)
	(4,-2)
	(4,-1)
	(5,0)
};
\addplot[white!70!black] fill between[of=A and B];
\end{axis}
\end{tikzpicture}
\noindent\begin{tikzpicture}
\begin{axis}[width = 4.5cm, ytick = {-2,-1,0,1,2}, xtick = {1,2,3,4,5}, xmin = 0.5, xmax = 5.500000, ymin = -2.5, ymax = 2.5, axis line style = {draw = none}, xmajorgrids, ymajorgrids, separate axis lines, yticklabels = {vb, b, n, g, vg}, xticklabels = {$c_1$,$c_2$,$c_3$,$c_4$,$c_5$}]
\addplot[name path = A, black, no markers, solid, very thick]
coordinates {
	(1,2)
	(2,0)
	(3,2)
	(4,-1)
	(5,0)
};
\addplot[name path = B, black, no markers, solid, very thick]
coordinates {
	(1,1)
	(2,-2)
	(3,2)
	(4,-2)
	(5,-2)
};
\addplot[name path = C, black, mark = *, only marks, very thick]
coordinates {
	(1,1)
	(1,2)
	(2,-2)
	(2,-1)
	(2,0)
	(3,2)
	(4,-2)
	(4,-1)
	(5,-2)
	(5,-1)
	(5,0)
};
\addplot[white!70!black] fill between[of=A and B];
\end{axis}
\end{tikzpicture}
\noindent\begin{tikzpicture}
\begin{axis}[width = 4.5cm, ytick = {-2,-1,0,1,2}, xtick = {1,2,3,4,5}, xmin = 0.5, xmax = 5.500000, ymin = -2.5, ymax = 2.5, axis line style = {draw = none}, xmajorgrids, ymajorgrids, separate axis lines, yticklabels = {vb, b, n, g, vg}, xticklabels = {$c_1$,$c_2$,$c_3$,$c_4$,$c_5$}]
\addplot[name path = A, black, no markers, solid, very thick]
coordinates {
	(1,2)
	(2,0)
	(3,2)
	(4,-1)
	(5,0)
};
\addplot[name path = B, black, no markers, solid, very thick]
coordinates {
	(1,1)
	(2,-1)
	(3,0)
	(4,-2)
	(5,-2)
};
\addplot[name path = C, black, mark = *, only marks, very thick]
coordinates {
	(1,1)
	(1,2)
	(2,-1)
	(2,0)
	(3,0)
	(3,1)
	(3,2)
	(4,-2)
	(4,-1)
	(5,-2)
	(5,-1)
	(5,0)
};
\addplot[white!70!black] fill between[of=A and B];
\end{axis}
\end{tikzpicture}
\noindent\begin{tikzpicture}
\begin{axis}[width = 4.5cm, ytick = {-2,-1,0,1,2}, xtick = {1,2,3,4,5}, xmin = 0.5, xmax = 5.500000, ymin = -2.5, ymax = 2.5, axis line style = {draw = none}, xmajorgrids, ymajorgrids, separate axis lines, yticklabels = {vb, b, n, g, vg}, xticklabels = {$c_1$,$c_2$,$c_3$,$c_4$,$c_5$}]
\addplot[name path = A, black, no markers, solid, very thick]
coordinates {
	(1,-1)
	(2,1)
	(3,2)
	(4,2)
	(5,2)
};
\addplot[name path = B, black, no markers, solid, very thick]
coordinates {
	(1,-2)
	(2,-1)
	(3,2)
	(4,0)
	(5,1)
};
\addplot[name path = C, black, mark = *, only marks, very thick]
coordinates {
	(1,-2)
	(1,-1)
	(2,-1)
	(2,0)
	(2,1)
	(3,2)
	(4,0)
	(4,1)
	(4,2)
	(5,1)
	(5,2)
};
\addplot[white!70!black] fill between[of=A and B];
\end{axis}
\end{tikzpicture}

\noindent\begin{tikzpicture}
\begin{axis}[width = 4.5cm, ytick = {-2,-1,0,1,2}, xtick = {1,2,3,4,5}, xmin = 0.5, xmax = 5.500000, ymin = -2.5, ymax = 2.5, axis line style = {draw = none}, xmajorgrids, ymajorgrids, separate axis lines, yticklabels = {vb, b, n, g, vg}, xticklabels = {$c_1$,$c_2$,$c_3$,$c_4$,$c_5$}]
\addplot[name path = A, black, no markers, solid, very thick]
coordinates {
	(1,0)
	(2,1)
	(3,-1)
	(4,2)
	(5,2)
};
\addplot[name path = B, black, no markers, solid, very thick]
coordinates {
	(1,0)
	(2,-2)
	(3,-2)
	(4,-2)
	(5,1)
};
\addplot[name path = C, black, mark = *, only marks, very thick]
coordinates {
	(1,0)
	(2,-2)
	(2,-1)
	(2,0)
	(2,1)
	(3,-2)
	(3,-1)
	(4,-2)
	(4,-1)
	(4,0)
	(4,1)
	(4,2)
	(5,1)
	(5,2)
};
\addplot[white!70!black] fill between[of=A and B];
\end{axis}
\end{tikzpicture}
\noindent\begin{tikzpicture}
\begin{axis}[width = 4.5cm, ytick = {-2,-1,0,1,2}, xtick = {1,2,3,4,5}, xmin = 0.5, xmax = 5.500000, ymin = -2.5, ymax = 2.5, axis line style = {draw = none}, xmajorgrids, ymajorgrids, separate axis lines, yticklabels = {vb, b, n, g, vg}, xticklabels = {$c_1$,$c_2$,$c_3$,$c_4$,$c_5$}]
\addplot[name path = A, black, no markers, solid, very thick]
coordinates {
	(1,-1)
	(2,2)
	(3,2)
	(4,2)
	(5,2)
};
\addplot[name path = B, black, no markers, solid, very thick]
coordinates {
	(1,-2)
	(2,2)
	(3,2)
	(4,0)
	(5,0)
};
\addplot[name path = C, black, mark = *, only marks, very thick]
coordinates {
	(1,-2)
	(1,-1)
	(2,2)
	(3,2)
	(4,0)
	(4,1)
	(4,2)
	(5,0)
	(5,1)
	(5,2)
};
\addplot[white!70!black] fill between[of=A and B];
\end{axis}
\end{tikzpicture}
\noindent\begin{tikzpicture}
\begin{axis}[width = 4.5cm, ytick = {-2,-1,0,1,2}, xtick = {1,2,3,4,5}, xmin = 0.5, xmax = 5.500000, ymin = -2.5, ymax = 2.5, axis line style = {draw = none}, xmajorgrids, ymajorgrids, separate axis lines, yticklabels = {vb, b, n, g, vg}, xticklabels = {$c_1$,$c_2$,$c_3$,$c_4$,$c_5$}]
\addplot[name path = A, black, no markers, solid, very thick]
coordinates {
	(1,2)
	(2,0)
	(3,-1)
	(4,1)
	(5,0)
};
\addplot[name path = B, black, no markers, solid, very thick]
coordinates {
	(1,1)
	(2,-2)
	(3,-2)
	(4,-2)
	(5,0)
};
\addplot[name path = C, black, mark = *, only marks, very thick]
coordinates {
	(1,1)
	(1,2)
	(2,-2)
	(2,-1)
	(2,0)
	(3,-2)
	(3,-1)
	(4,-2)
	(4,-1)
	(4,0)
	(4,1)
	(5,0)
};
\addplot[white!70!black] fill between[of=A and B];
\end{axis}
\end{tikzpicture}
\noindent\begin{tikzpicture}
\begin{axis}[width = 4.5cm, ytick = {-2,-1,0,1,2}, xtick = {1,2,3,4,5}, xmin = 0.5, xmax = 5.500000, ymin = -2.5, ymax = 2.5, axis line style = {draw = none}, xmajorgrids, ymajorgrids, separate axis lines, yticklabels = {vb, b, n, g, vg}, xticklabels = {$c_1$,$c_2$,$c_3$,$c_4$,$c_5$}]
\addplot[name path = A, black, no markers, solid, very thick]
coordinates {
	(1,2)
	(2,0)
	(3,-1)
	(4,1)
	(5,0)
};
\addplot[name path = B, black, no markers, solid, very thick]
coordinates {
	(1,1)
	(2,-2)
	(3,-2)
	(4,0)
	(5,-2)
};
\addplot[name path = C, black, mark = *, only marks, very thick]
coordinates {
	(1,1)
	(1,2)
	(2,-2)
	(2,-1)
	(2,0)
	(3,-2)
	(3,-1)
	(4,0)
	(4,1)
	(5,-2)
	(5,-1)
	(5,0)
};
\addplot[white!70!black] fill between[of=A and B];
\end{axis}
\end{tikzpicture}

\noindent\begin{tikzpicture}
\begin{axis}[width = 4.5cm, ytick = {-2,-1,0,1,2}, xtick = {1,2,3,4,5}, xmin = 0.5, xmax = 5.500000, ymin = -2.5, ymax = 2.5, axis line style = {draw = none}, xmajorgrids, ymajorgrids, separate axis lines, yticklabels = {vb, b, n, g, vg}, xticklabels = {$c_1$,$c_2$,$c_3$,$c_4$,$c_5$}]
\addplot[name path = A, black, no markers, solid, very thick]
coordinates {
	(1,2)
	(2,0)
	(3,-1)
	(4,1)
	(5,0)
};
\addplot[name path = B, black, no markers, solid, very thick]
coordinates {
	(1,1)
	(2,-1)
	(3,-2)
	(4,-2)
	(5,-2)
};
\addplot[name path = C, black, mark = *, only marks, very thick]
coordinates {
	(1,1)
	(1,2)
	(2,-1)
	(2,0)
	(3,-2)
	(3,-1)
	(4,-2)
	(4,-1)
	(4,0)
	(4,1)
	(5,-2)
	(5,-1)
	(5,0)
};
\addplot[white!70!black] fill between[of=A and B];
\end{axis}
\end{tikzpicture}
	
		\vspace{1ex}
		\hrule
		\vspace{1ex}
		
	{\bf Bad Contributors}
	
		\vspace{1ex}
		\hrule
		\vspace{1ex}
	
\noindent\begin{tikzpicture}
\begin{axis}[width = 4.5cm, ytick = {-2,-1,0,1,2}, xtick = {1,2,3,4,5}, xmin = 0.5, xmax = 5.500000, ymin = -2.5, ymax = 2.5, axis line style = {draw = none}, xmajorgrids, ymajorgrids, separate axis lines, yticklabels = {vb, b, n, g, vg}, xticklabels = {$c_1$,$c_2$,$c_3$,$c_4$,$c_5$}]
\addplot[name path = A, black, no markers, solid, very thick]
coordinates {
	(1,2)
	(2,-2)
	(3,1)
	(4,-1)
	(5,-1)
};
\addplot[name path = B, black, no markers, solid, very thick]
coordinates {
	(1,-2)
	(2,-2)
	(3,-2)
	(4,-2)
	(5,-2)
};
\addplot[name path = C, black, mark = *, only marks, very thick]
coordinates {
	(1,-2)
	(1,-1)
	(1,0)
	(1,1)
	(1,2)
	(2,-2)
	(3,-2)
	(3,-1)
	(3,0)
	(3,1)
	(4,-2)
	(4,-1)
	(5,-2)
	(5,-1)
};
\addplot[white!70!black] fill between[of=A and B];
\end{axis}
\end{tikzpicture}
\noindent\begin{tikzpicture}
\begin{axis}[width = 4.5cm, ytick = {-2,-1,0,1,2}, xtick = {1,2,3,4,5}, xmin = 0.5, xmax = 5.500000, ymin = -2.5, ymax = 2.5, axis line style = {draw = none}, xmajorgrids, ymajorgrids, separate axis lines, yticklabels = {vb, b, n, g, vg}, xticklabels = {$c_1$,$c_2$,$c_3$,$c_4$,$c_5$}]
\addplot[name path = A, black, no markers, solid, very thick]
coordinates {
	(1,-1)
	(2,1)
	(3,2)
	(4,2)
	(5,0)
};
\addplot[name path = B, black, no markers, solid, very thick]
coordinates {
	(1,-2)
	(2,-2)
	(3,-2)
	(4,-2)
	(5,-2)
};
\addplot[name path = C, black, mark = *, only marks, very thick]
coordinates {
	(1,-2)
	(1,-1)
	(2,-2)
	(2,-1)
	(2,0)
	(2,1)
	(3,-2)
	(3,-1)
	(3,0)
	(3,1)
	(3,2)
	(4,-2)
	(4,-1)
	(4,0)
	(4,1)
	(4,2)
	(5,-2)
	(5,-1)
	(5,0)
};
\addplot[white!70!black] fill between[of=A and B];
\end{axis}
\end{tikzpicture}
\noindent\begin{tikzpicture}
\begin{axis}[width = 4.5cm, ytick = {-2,-1,0,1,2}, xtick = {1,2,3,4,5}, xmin = 0.5, xmax = 5.500000, ymin = -2.5, ymax = 2.5, axis line style = {draw = none}, xmajorgrids, ymajorgrids, separate axis lines, yticklabels = {vb, b, n, g, vg}, xticklabels = {$c_1$,$c_2$,$c_3$,$c_4$,$c_5$}]
\addplot[name path = A, black, no markers, solid, very thick]
coordinates {
	(1,0)
	(2,1)
	(3,-1)
	(4,2)
	(5,0)
};
\addplot[name path = B, black, no markers, solid, very thick]
coordinates {
	(1,-2)
	(2,-2)
	(3,-2)
	(4,-2)
	(5,-2)
};
\addplot[name path = C, black, mark = *, only marks, very thick]
coordinates {
	(1,-2)
	(1,-1)
	(1,0)
	(2,-2)
	(2,-1)
	(2,0)
	(2,1)
	(3,-2)
	(3,-1)
	(4,-2)
	(4,-1)
	(4,0)
	(4,1)
	(4,2)
	(5,-2)
	(5,-1)
	(5,0)
};
\addplot[white!70!black] fill between[of=A and B];
\end{axis}
\end{tikzpicture}
	
		\vspace{1ex}
		\hrule
	
    \caption{Assignment rules for the final model of \textsc{CM2};}\label{fig:ex2-rules1}
\end{figure}

The community manager carefully inspected the boundary profiles, focusing on the top and bottom categories, since the rules from the middle category were simply the complement of all of the other rules.

The manager decided to tweak the first two rules from the Good category by raising boundary on the second criterion from $very\ bad$ to $bad$, as neutrally being committed to a project and having a clear inability to work with others would not be characteristic of a good contributor, regardless of all other evaluations. The remaining three rules from the top category stayed unchanged.

When looking at the rules for the bottom category, the community manager found that a very committed contributor should not be in the bottom category, despite other poor evaluations on other criteria. He mentioned that if the commitment were to fall to a neutral evaluation, this profile would indeed be a bad contributor. Additionally, the second and third rules from the Bad category were also adjusted by lowering slightly the good evaluation on the second criterion to a neutral one. The final model is illustrated in Fig.~\ref{fig:ex2-model4}, while the assignment rules derived from it are presented in Fig.~\ref{fig:ex2-rules2}.

\begin{figure}
\centering

\begin{minipage}[c]{0.4\columnwidth}
\vspace{0pt}
\begin{tikzpicture}
\begin{axis}[title ={Bad-Neutral}, height=\textwidth,width=\textwidth, xmin = 0.5, xmax = 5.5, ymin = -2.5, ymax = 2.5, every axis x label/.style={at={(ticklabel* cs:0.97)},anchor=south},xtick={1,2,3,4,5}, xticklabels={$c_1$,$c_2$,$c_3$,$c_4$,$c_5$}, ytick={-2,-1,0,1,2}, xmajorgrids = true, axis line style = { draw = none }, yticklabels = {vb, b, n, g, vg}, ymajorgrids = true]
\addplot[name path=T]
coordinates {
	(1,3)
	(2,3)
	(3,3)
	(4,3)
	(5,3)
};
\addplot[name path=B]
coordinates {
	(1,-3)
	(2,-3)
	(3,-3)
	(4,-3)
	(5,-3)
};
\addplot[name path=P, black, solid, mark = *]
coordinates {
	(1,-0.5)
	(2,-1.5)
	(3,-0.5)
	(4,0.5)
	(5,0.5)
};
\addplot[name path=V]
coordinates {
	(1,-1)
	(2,-2)
	(3,-1)
	(4,-3)
	(5,-3)
};
\addplot[name path=D]
coordinates {
	(1,1)
	(2,1)
	(3,3)
	(4,2)
	(5,1)
};
\addplot[black] fill between[of=V and B];
\addplot[pattern = north east lines] fill between[of=D and T];
\end{axis}
\end{tikzpicture}
\end{minipage}
\begin{minipage}[c]{0.4\columnwidth}
\vspace{0pt}

\begin{tikzpicture}
\begin{axis}[title ={Neutral-Good}, height=\textwidth,width=\textwidth, xmin = 0.5, xmax = 5.5, ymin = -2.5, ymax = 2.5, every axis x label/.style={at={(ticklabel* cs:0.97)},anchor=south},xtick={1,2,3,4,5}, xticklabels={$c_1$,$c_2$,$c_3$,$c_4$,$c_5$}, ytick={-2,-1,0,1,2}, xmajorgrids = true, axis line style = { draw = none }, yticklabels = {vb, b, n, g, vg}, ymajorgrids = true]
\addplot[name path=T]
coordinates {
	(1,3)
	(2,3)
	(3,3)
	(4,3)
	(5,3)
};
\addplot[name path=B]
coordinates {
	(1,-3)
	(2,-3)
	(3,-3)
	(4,-3)
	(5,-3)
};
\addplot[name path=P, black, solid, mark = *]
coordinates {
	(1,-0.5)
	(2,0.5)
	(3,2.5)
	(4,1.5)
	(5,0.5)
};
\addplot[name path=V]
coordinates {
	(1,-1)
	(2,-2)
	(3,-1)
	(4,-2)
	(5,-3)
};
\addplot[name path=D]
coordinates {
	(1,1)
	(2,2)
	(3,3)
	(4,3)
	(5,3)
};
\addplot[black] fill between[of=V and B];
\addplot[pattern = north east lines] fill between[of=D and T];
\end{axis}
\end{tikzpicture}
\end{minipage}
\begin{minipage}[c]{0.07\columnwidth}
\vspace{0pt}

\begin{tabular}{c|c}
$\lambda$ & 0.55 \\\hline
$c_1$ & 0.40 \\
$c_2$ & 0.15 \\
$c_3$ & 0.15 \\
$c_4$ & 0.15 \\
$c_5$ & 0.15
\end{tabular}
\end{minipage}

\caption{Final preference model of \DB (MR-Sort with vetoes weakened by dictators).}\label{fig:ex2-model4}
\end{figure}

\begin{figure}
	\centering
	
		\hrule
		\vspace{1ex}
	
	{\bf Good Contributors}
	
		\vspace{1ex}
		\hrule
		\vspace{1ex}

\noindent\begin{tikzpicture}
\begin{axis}[width = 4.5cm, ytick = {-2,-1,0,1,2}, xtick = {1,2,3,4,5}, xmin = 0.5, xmax = 5.500000, ymin = -2.5, ymax = 2.5, axis line style = {draw = none}, xmajorgrids, ymajorgrids, separate axis lines, yticklabels = {vb, b, n, g, vg}, xticklabels = {$c_1$,$c_2$,$c_3$,$c_4$,$c_5$}]
\addplot[name path = A, black, no markers, solid, very thick]
coordinates {
	(1,2)
	(2,2)
	(3,2)
	(4,2)
	(5,2)
};
\addplot[name path = B, black, no markers, solid, very thick]
coordinates {
	(1,0)
	(2,-1)
	(3,0)
	(4,-1)
	(5,1)
};
\addplot[name path = C, black, mark = *, only marks, very thick]
coordinates {
	(1,0)
	(1,1)
	(1,2)
	(2,-1)
	(2,0)
	(2,1)
	(2,2)
	(3,0)
	(3,1)
	(3,2)
	(4,-1)
	(4,0)
	(4,1)
	(4,2)
	(5,1)
	(5,2)
};
\addplot[white!70!black] fill between[of=A and B];
\end{axis}
\end{tikzpicture}
\noindent\begin{tikzpicture}
\begin{axis}[width = 4.5cm, ytick = {-2,-1,0,1,2}, xtick = {1,2,3,4,5}, xmin = 0.5, xmax = 5.500000, ymin = -2.5, ymax = 2.5, axis line style = {draw = none}, xmajorgrids, ymajorgrids, separate axis lines, yticklabels = {vb, b, n, g, vg}, xticklabels = {$c_1$,$c_2$,$c_3$,$c_4$,$c_5$}]
\addplot[name path = A, black, no markers, solid, very thick]
coordinates {
	(1,2)
	(2,2)
	(3,2)
	(4,2)
	(5,2)
};
\addplot[name path = B, black, no markers, solid, very thick]
coordinates {
	(1,0)
	(2,-1)
	(3,0)
	(4,2)
	(5,-2)
};
\addplot[name path = C, black, mark = *, only marks, very thick]
coordinates {
	(1,0)
	(1,1)
	(1,2)
	(2,-1)
	(2,0)
	(2,1)
	(2,2)
	(3,0)
	(3,1)
	(3,2)
	(4,2)
	(5,-2)
	(5,-1)
	(5,0)
	(5,1)
	(5,2)
};
\addplot[white!70!black] fill between[of=A and B];
\end{axis}
\end{tikzpicture}
\noindent\begin{tikzpicture}
\begin{axis}[width = 4.5cm, ytick = {-2,-1,0,1,2}, xtick = {1,2,3,4,5}, xmin = 0.5, xmax = 5.500000, ymin = -2.5, ymax = 2.5, axis line style = {draw = none}, xmajorgrids, ymajorgrids, separate axis lines, yticklabels = {vb, b, n, g, vg}, xticklabels = {$c_1$,$c_2$,$c_3$,$c_4$,$c_5$}]
\addplot[name path = A, black, no markers, solid, very thick]
coordinates {
	(1,2)
	(2,2)
	(3,2)
	(4,2)
	(5,2)
};
\addplot[name path = B, black, no markers, solid, very thick]
coordinates {
	(1,0)
	(2,1)
	(3,0)
	(4,-1)
	(5,-2)
};
\addplot[name path = C, black, mark = *, only marks, very thick]
coordinates {
	(1,0)
	(1,1)
	(1,2)
	(2,1)
	(2,2)
	(3,0)
	(3,1)
	(3,2)
	(4,-1)
	(4,0)
	(4,1)
	(4,2)
	(5,-2)
	(5,-1)
	(5,0)
	(5,1)
	(5,2)
};
\addplot[white!70!black] fill between[of=A and B];
\end{axis}
\end{tikzpicture}
\noindent\begin{tikzpicture}
\begin{axis}[width = 4.5cm, ytick = {-2,-1,0,1,2}, xtick = {1,2,3,4,5}, xmin = 0.5, xmax = 5.500000, ymin = -2.5, ymax = 2.5, axis line style = {draw = none}, xmajorgrids, ymajorgrids, separate axis lines, yticklabels = {vb, b, n, g, vg}, xticklabels = {$c_1$,$c_2$,$c_3$,$c_4$,$c_5$}]
\addplot[name path = A, black, no markers, solid, very thick]
coordinates {
	(1,2)
	(2,2)
	(3,2)
	(4,2)
	(5,2)
};
\addplot[name path = B, black, no markers, solid, very thick]
coordinates {
	(1,1)
	(2,-2)
	(3,-2)
	(4,-2)
	(5,1)
};
\addplot[name path = C, black, mark = *, only marks, very thick]
coordinates {
	(1,1)
	(1,2)
	(2,-2)
	(2,-1)
	(2,0)
	(2,1)
	(2,2)
	(3,-2)
	(3,-1)
	(3,0)
	(3,1)
	(3,2)
	(4,-2)
	(4,-1)
	(4,0)
	(4,1)
	(4,2)
	(5,1)
	(5,2)
};
\addplot[white!70!black] fill between[of=A and B];
\end{axis}
\end{tikzpicture}

\noindent\begin{tikzpicture}
\begin{axis}[width = 4.5cm, ytick = {-2,-1,0,1,2}, xtick = {1,2,3,4,5}, xmin = 0.5, xmax = 5.500000, ymin = -2.5, ymax = 2.5, axis line style = {draw = none}, xmajorgrids, ymajorgrids, separate axis lines, yticklabels = {vb, b, n, g, vg}, xticklabels = {$c_1$,$c_2$,$c_3$,$c_4$,$c_5$}]
\addplot[name path = A, black, no markers, solid, very thick]
coordinates {
	(1,2)
	(2,2)
	(3,2)
	(4,2)
	(5,2)
};
\addplot[name path = B, black, no markers, solid, very thick]
coordinates {
	(1,0)
	(2,2)
	(3,-2)
	(4,-2)
	(5,-2)
};
\addplot[name path = C, black, mark = *, only marks, very thick]
coordinates {
	(1,0)
	(1,1)
	(1,2)
	(2,2)
	(3,-2)
	(3,-1)
	(3,0)
	(3,1)
	(3,2)
	(4,-2)
	(4,-1)
	(4,0)
	(4,1)
	(4,2)
	(5,-2)
	(5,-1)
	(5,0)
	(5,1)
	(5,2)
};
\addplot[white!70!black] fill between[of=A and B];
\end{axis}
\end{tikzpicture}
	
		\vspace{1ex}
		\hrule
		\vspace{1ex}
	
	{\bf Neutral Contributors}
	
		\vspace{1ex}
		\hrule
		\vspace{1ex}
	
\noindent\begin{tikzpicture}
\begin{axis}[width = 4.5cm, ytick = {-2,-1,0,1,2}, xtick = {1,2,3,4,5}, xmin = 0.5, xmax = 5.500000, ymin = -2.5, ymax = 2.5, axis line style = {draw = none}, xmajorgrids, ymajorgrids, separate axis lines, yticklabels = {vb, b, n, g, vg}, xticklabels = {$c_1$,$c_2$,$c_3$,$c_4$,$c_5$}]
\addplot[name path = A, black, no markers, solid, very thick]
coordinates {
	(1,2)
	(2,0)
	(3,2)
	(4,1)
	(5,0)
};
\addplot[name path = B, black, no markers, solid, very thick]
coordinates {
	(1,0)
	(2,-1)
	(3,0)
	(4,-1)
	(5,-2)
};
\addplot[name path = C, black, mark = *, only marks, very thick]
coordinates {
	(1,0)
	(1,1)
	(1,2)
	(2,-1)
	(2,0)
	(3,0)
	(3,1)
	(3,2)
	(4,-1)
	(4,0)
	(4,1)
	(5,-2)
	(5,-1)
	(5,0)
};
\addplot[white!70!black] fill between[of=A and B];
\end{axis}
\end{tikzpicture}
\noindent\begin{tikzpicture}
\begin{axis}[width = 4.5cm, ytick = {-2,-1,0,1,2}, xtick = {1,2,3,4,5}, xmin = 0.5, xmax = 5.500000, ymin = -2.5, ymax = 2.5, axis line style = {draw = none}, xmajorgrids, ymajorgrids, separate axis lines, yticklabels = {vb, b, n, g, vg}, xticklabels = {$c_1$,$c_2$,$c_3$,$c_4$,$c_5$}]
\addplot[name path = A, black, no markers, solid, very thick]
coordinates {
	(1,0)
	(2,1)
	(3,2)
	(4,-2)
	(5,2)
};
\addplot[name path = B, black, no markers, solid, very thick]
coordinates {
	(1,0)
	(2,-1)
	(3,0)
	(4,-2)
	(5,-2)
};
\addplot[name path = C, black, mark = *, only marks, very thick]
coordinates {
	(1,0)
	(2,-1)
	(2,0)
	(2,1)
	(3,0)
	(3,1)
	(3,2)
	(4,-2)
	(5,-2)
	(5,-1)
	(5,0)
	(5,1)
	(5,2)
};
\addplot[white!70!black] fill between[of=A and B];
\end{axis}
\end{tikzpicture}
\noindent\begin{tikzpicture}
\begin{axis}[width = 4.5cm, ytick = {-2,-1,0,1,2}, xtick = {1,2,3,4,5}, xmin = 0.5, xmax = 5.500000, ymin = -2.5, ymax = 2.5, axis line style = {draw = none}, xmajorgrids, ymajorgrids, separate axis lines, yticklabels = {vb, b, n, g, vg}, xticklabels = {$c_1$,$c_2$,$c_3$,$c_4$,$c_5$}]
\addplot[name path = A, black, no markers, solid, very thick]
coordinates {
	(1,2)
	(2,0)
	(3,2)
	(4,-2)
	(5,0)
};
\addplot[name path = B, black, no markers, solid, very thick]
coordinates {
	(1,1)
	(2,-1)
	(3,0)
	(4,-2)
	(5,-2)
};
\addplot[name path = C, black, mark = *, only marks, very thick]
coordinates {
	(1,1)
	(1,2)
	(2,-1)
	(2,0)
	(3,0)
	(3,1)
	(3,2)
	(4,-2)
	(5,-2)
	(5,-1)
	(5,0)
};
\addplot[white!70!black] fill between[of=A and B];
\end{axis}
\end{tikzpicture}
\noindent\begin{tikzpicture}
\begin{axis}[width = 4.5cm, ytick = {-2,-1,0,1,2}, xtick = {1,2,3,4,5}, xmin = 0.5, xmax = 5.500000, ymin = -2.5, ymax = 2.5, axis line style = {draw = none}, xmajorgrids, ymajorgrids, separate axis lines, yticklabels = {vb, b, n, g, vg}, xticklabels = {$c_1$,$c_2$,$c_3$,$c_4$,$c_5$}]
\addplot[name path = A, black, no markers, solid, very thick]
coordinates {
	(1,-1)
	(2,1)
	(3,2)
	(4,2)
	(5,2)
};
\addplot[name path = B, black, no markers, solid, very thick]
coordinates {
	(1,-2)
	(2,-1)
	(3,0)
	(4,1)
	(5,1)
};
\addplot[name path = C, black, mark = *, only marks, very thick]
coordinates {
	(1,-2)
	(1,-1)
	(2,-1)
	(2,0)
	(2,1)
	(3,0)
	(3,1)
	(3,2)
	(4,1)
	(4,2)
	(5,1)
	(5,2)
};
\addplot[white!70!black] fill between[of=A and B];
\end{axis}
\end{tikzpicture}

\noindent\begin{tikzpicture}
\begin{axis}[width = 4.5cm, ytick = {-2,-1,0,1,2}, xtick = {1,2,3,4,5}, xmin = 0.5, xmax = 5.500000, ymin = -2.5, ymax = 2.5, axis line style = {draw = none}, xmajorgrids, ymajorgrids, separate axis lines, yticklabels = {vb, b, n, g, vg}, xticklabels = {$c_1$,$c_2$,$c_3$,$c_4$,$c_5$}]
\addplot[name path = A, black, no markers, solid, very thick]
coordinates {
	(1,0)
	(2,-2)
	(3,2)
	(4,2)
	(5,2)
};
\addplot[name path = B, black, no markers, solid, very thick]
coordinates {
	(1,0)
	(2,-2)
	(3,-2)
	(4,-2)
	(5,1)
};
\addplot[name path = C, black, mark = *, only marks, very thick]
coordinates {
	(1,0)
	(2,-2)
	(3,-2)
	(3,-1)
	(3,0)
	(3,1)
	(3,2)
	(4,-2)
	(4,-1)
	(4,0)
	(4,1)
	(4,2)
	(5,1)
	(5,2)
};
\addplot[white!70!black] fill between[of=A and B];
\end{axis}
\end{tikzpicture}
\noindent\begin{tikzpicture}
\begin{axis}[width = 4.5cm, ytick = {-2,-1,0,1,2}, xtick = {1,2,3,4,5}, xmin = 0.5, xmax = 5.500000, ymin = -2.5, ymax = 2.5, axis line style = {draw = none}, xmajorgrids, ymajorgrids, separate axis lines, yticklabels = {vb, b, n, g, vg}, xticklabels = {$c_1$,$c_2$,$c_3$,$c_4$,$c_5$}]
\addplot[name path = A, black, no markers, solid, very thick]
coordinates {
	(1,0)
	(2,1)
	(3,-1)
	(4,2)
	(5,2)
};
\addplot[name path = B, black, no markers, solid, very thick]
coordinates {
	(1,0)
	(2,-2)
	(3,-2)
	(4,-2)
	(5,1)
};
\addplot[name path = C, black, mark = *, only marks, very thick]
coordinates {
	(1,0)
	(2,-2)
	(2,-1)
	(2,0)
	(2,1)
	(3,-2)
	(3,-1)
	(4,-2)
	(4,-1)
	(4,0)
	(4,1)
	(4,2)
	(5,1)
	(5,2)
};
\addplot[white!70!black] fill between[of=A and B];
\end{axis}
\end{tikzpicture}
\noindent\begin{tikzpicture}
\begin{axis}[width = 4.5cm, ytick = {-2,-1,0,1,2}, xtick = {1,2,3,4,5}, xmin = 0.5, xmax = 5.500000, ymin = -2.5, ymax = 2.5, axis line style = {draw = none}, xmajorgrids, ymajorgrids, separate axis lines, yticklabels = {vb, b, n, g, vg}, xticklabels = {$c_1$,$c_2$,$c_3$,$c_4$,$c_5$}]
\addplot[name path = A, black, no markers, solid, very thick]
coordinates {
	(1,0)
	(2,-2)
	(3,2)
	(4,2)
	(5,2)
};
\addplot[name path = B, black, no markers, solid, very thick]
coordinates {
	(1,0)
	(2,-2)
	(3,-2)
	(4,2)
	(5,-2)
};
\addplot[name path = C, black, mark = *, only marks, very thick]
coordinates {
	(1,0)
	(2,-2)
	(3,-2)
	(3,-1)
	(3,0)
	(3,1)
	(3,2)
	(4,2)
	(5,-2)
	(5,-1)
	(5,0)
	(5,1)
	(5,2)
};
\addplot[white!70!black] fill between[of=A and B];
\end{axis}
\end{tikzpicture}
\noindent\begin{tikzpicture}
\begin{axis}[width = 4.5cm, ytick = {-2,-1,0,1,2}, xtick = {1,2,3,4,5}, xmin = 0.5, xmax = 5.500000, ymin = -2.5, ymax = 2.5, axis line style = {draw = none}, xmajorgrids, ymajorgrids, separate axis lines, yticklabels = {vb, b, n, g, vg}, xticklabels = {$c_1$,$c_2$,$c_3$,$c_4$,$c_5$}]
\addplot[name path = A, black, no markers, solid, very thick]
coordinates {
	(1,0)
	(2,1)
	(3,-1)
	(4,2)
	(5,2)
};
\addplot[name path = B, black, no markers, solid, very thick]
coordinates {
	(1,0)
	(2,-2)
	(3,-2)
	(4,2)
	(5,-2)
};
\addplot[name path = C, black, mark = *, only marks, very thick]
coordinates {
	(1,0)
	(2,-2)
	(2,-1)
	(2,0)
	(2,1)
	(3,-2)
	(3,-1)
	(4,2)
	(5,-2)
	(5,-1)
	(5,0)
	(5,1)
	(5,2)
};
\addplot[white!70!black] fill between[of=A and B];
\end{axis}
\end{tikzpicture}

\noindent\begin{tikzpicture}
\begin{axis}[width = 4.5cm, ytick = {-2,-1,0,1,2}, xtick = {1,2,3,4,5}, xmin = 0.5, xmax = 5.500000, ymin = -2.5, ymax = 2.5, axis line style = {draw = none}, xmajorgrids, ymajorgrids, separate axis lines, yticklabels = {vb, b, n, g, vg}, xticklabels = {$c_1$,$c_2$,$c_3$,$c_4$,$c_5$}]
\addplot[name path = A, black, no markers, solid, very thick]
coordinates {
	(1,0)
	(2,1)
	(3,-1)
	(4,2)
	(5,2)
};
\addplot[name path = B, black, no markers, solid, very thick]
coordinates {
	(1,0)
	(2,1)
	(3,-2)
	(4,-2)
	(5,-2)
};
\addplot[name path = C, black, mark = *, only marks, very thick]
coordinates {
	(1,0)
	(2,1)
	(3,-2)
	(3,-1)
	(4,-2)
	(4,-1)
	(4,0)
	(4,1)
	(4,2)
	(5,-2)
	(5,-1)
	(5,0)
	(5,1)
	(5,2)
};
\addplot[white!70!black] fill between[of=A and B];
\end{axis}
\end{tikzpicture}
\noindent\begin{tikzpicture}
\begin{axis}[width = 4.5cm, ytick = {-2,-1,0,1,2}, xtick = {1,2,3,4,5}, xmin = 0.5, xmax = 5.500000, ymin = -2.5, ymax = 2.5, axis line style = {draw = none}, xmajorgrids, ymajorgrids, separate axis lines, yticklabels = {vb, b, n, g, vg}, xticklabels = {$c_1$,$c_2$,$c_3$,$c_4$,$c_5$}]
\addplot[name path = A, black, no markers, solid, very thick]
coordinates {
	(1,-1)
	(2,2)
	(3,2)
	(4,2)
	(5,2)
};
\addplot[name path = B, black, no markers, solid, very thick]
coordinates {
	(1,-2)
	(2,2)
	(3,0)
	(4,1)
	(5,1)
};
\addplot[name path = C, black, mark = *, only marks, very thick]
coordinates {
	(1,-2)
	(1,-1)
	(2,2)
	(3,0)
	(3,1)
	(3,2)
	(4,1)
	(4,2)
	(5,1)
	(5,2)
};
\addplot[white!70!black] fill between[of=A and B];
\end{axis}
\end{tikzpicture}
\noindent\begin{tikzpicture}
\begin{axis}[width = 4.5cm, ytick = {-2,-1,0,1,2}, xtick = {1,2,3,4,5}, xmin = 0.5, xmax = 5.500000, ymin = -2.5, ymax = 2.5, axis line style = {draw = none}, xmajorgrids, ymajorgrids, separate axis lines, yticklabels = {vb, b, n, g, vg}, xticklabels = {$c_1$,$c_2$,$c_3$,$c_4$,$c_5$}]
\addplot[name path = A, black, no markers, solid, very thick]
coordinates {
	(1,2)
	(2,-2)
	(3,2)
	(4,1)
	(5,0)
};
\addplot[name path = B, black, no markers, solid, very thick]
coordinates {
	(1,1)
	(2,-2)
	(3,-2)
	(4,1)
	(5,-2)
};
\addplot[name path = C, black, mark = *, only marks, very thick]
coordinates {
	(1,1)
	(1,2)
	(2,-2)
	(3,-2)
	(3,-1)
	(3,0)
	(3,1)
	(3,2)
	(4,1)
	(5,-2)
	(5,-1)
	(5,0)
};
\addplot[white!70!black] fill between[of=A and B];
\end{axis}
\end{tikzpicture}
\noindent\begin{tikzpicture}
\begin{axis}[width = 4.5cm, ytick = {-2,-1,0,1,2}, xtick = {1,2,3,4,5}, xmin = 0.5, xmax = 5.500000, ymin = -2.5, ymax = 2.5, axis line style = {draw = none}, xmajorgrids, ymajorgrids, separate axis lines, yticklabels = {vb, b, n, g, vg}, xticklabels = {$c_1$,$c_2$,$c_3$,$c_4$,$c_5$}]
\addplot[name path = A, black, no markers, solid, very thick]
coordinates {
	(1,2)
	(2,0)
	(3,-1)
	(4,1)
	(5,0)
};
\addplot[name path = B, black, no markers, solid, very thick]
coordinates {
	(1,1)
	(2,-2)
	(3,-2)
	(4,1)
	(5,-2)
};
\addplot[name path = C, black, mark = *, only marks, very thick]
coordinates {
	(1,1)
	(1,2)
	(2,-2)
	(2,-1)
	(2,0)
	(3,-2)
	(3,-1)
	(4,1)
	(5,-2)
	(5,-1)
	(5,0)
};
\addplot[white!70!black] fill between[of=A and B];
\end{axis}
\end{tikzpicture}

\noindent\begin{tikzpicture}
\begin{axis}[width = 4.5cm, ytick = {-2,-1,0,1,2}, xtick = {1,2,3,4,5}, xmin = 0.5, xmax = 5.500000, ymin = -2.5, ymax = 2.5, axis line style = {draw = none}, xmajorgrids, ymajorgrids, separate axis lines, yticklabels = {vb, b, n, g, vg}, xticklabels = {$c_1$,$c_2$,$c_3$,$c_4$,$c_5$}]
\addplot[name path = A, black, no markers, solid, very thick]
coordinates {
	(1,2)
	(2,-2)
	(3,2)
	(4,1)
	(5,0)
};
\addplot[name path = B, black, no markers, solid, very thick]
coordinates {
	(1,1)
	(2,-2)
	(3,0)
	(4,-2)
	(5,-2)
};
\addplot[name path = C, black, mark = *, only marks, very thick]
coordinates {
	(1,1)
	(1,2)
	(2,-2)
	(3,0)
	(3,1)
	(3,2)
	(4,-2)
	(4,-1)
	(4,0)
	(4,1)
	(5,-2)
	(5,-1)
	(5,0)
};
\addplot[white!70!black] fill between[of=A and B];
\end{axis}
\end{tikzpicture}
\noindent\begin{tikzpicture}
\begin{axis}[width = 4.5cm, ytick = {-2,-1,0,1,2}, xtick = {1,2,3,4,5}, xmin = 0.5, xmax = 5.500000, ymin = -2.5, ymax = 2.5, axis line style = {draw = none}, xmajorgrids, ymajorgrids, separate axis lines, yticklabels = {vb, b, n, g, vg}, xticklabels = {$c_1$,$c_2$,$c_3$,$c_4$,$c_5$}]
\addplot[name path = A, black, no markers, solid, very thick]
coordinates {
	(1,2)
	(2,0)
	(3,-1)
	(4,1)
	(5,0)
};
\addplot[name path = B, black, no markers, solid, very thick]
coordinates {
	(1,1)
	(2,-1)
	(3,-2)
	(4,-2)
	(5,-2)
};
\addplot[name path = C, black, mark = *, only marks, very thick]
coordinates {
	(1,1)
	(1,2)
	(2,-1)
	(2,0)
	(3,-2)
	(3,-1)
	(4,-2)
	(4,-1)
	(4,0)
	(4,1)
	(5,-2)
	(5,-1)
	(5,0)
};
\addplot[white!70!black] fill between[of=A and B];
\end{axis}
\end{tikzpicture}
	
		\vspace{1ex}
		\hrule
		\vspace{1ex}
		
	{\bf Bad Contributors}
	
		\vspace{1ex}
		\hrule
		\vspace{1ex}
	
\noindent\begin{tikzpicture}
\begin{axis}[width = 4.5cm, ytick = {-2,-1,0,1,2}, xtick = {1,2,3,4,5}, xmin = 0.5, xmax = 5.500000, ymin = -2.5, ymax = 2.5, axis line style = {draw = none}, xmajorgrids, ymajorgrids, separate axis lines, yticklabels = {vb, b, n, g, vg}, xticklabels = {$c_1$,$c_2$,$c_3$,$c_4$,$c_5$}]
\addplot[name path = A, black, no markers, solid, very thick]
coordinates {
	(1,-1)
	(2,0)
	(3,2)
	(4,1)
	(5,0)
};
\addplot[name path = B, black, no markers, solid, very thick]
coordinates {
	(1,-2)
	(2,-2)
	(3,-2)
	(4,-2)
	(5,-2)
};
\addplot[name path = C, black, mark = *, only marks, very thick]
coordinates {
	(1,-2)
	(1,-1)
	(2,-2)
	(2,-1)
	(2,0)
	(3,-2)
	(3,-1)
	(3,0)
	(3,1)
	(3,2)
	(4,-2)
	(4,-1)
	(4,0)
	(4,1)
	(5,-2)
	(5,-1)
	(5,0)
};
\addplot[white!70!black] fill between[of=A and B];
\end{axis}
\end{tikzpicture}
\noindent\begin{tikzpicture}
\begin{axis}[width = 4.5cm, ytick = {-2,-1,0,1,2}, xtick = {1,2,3,4,5}, xmin = 0.5, xmax = 5.500000, ymin = -2.5, ymax = 2.5, axis line style = {draw = none}, xmajorgrids, ymajorgrids, separate axis lines, yticklabels = {vb, b, n, g, vg}, xticklabels = {$c_1$,$c_2$,$c_3$,$c_4$,$c_5$}]
\addplot[name path = A, black, no markers, solid, very thick]
coordinates {
	(1,0)
	(2,-2)
	(3,2)
	(4,1)
	(5,0)
};
\addplot[name path = B, black, no markers, solid, very thick]
coordinates {
	(1,-2)
	(2,-2)
	(3,-2)
	(4,-2)
	(5,-2)
};
\addplot[name path = C, black, mark = *, only marks, very thick]
coordinates {
	(1,-2)
	(1,-1)
	(1,0)
	(2,-2)
	(3,-2)
	(3,-1)
	(3,0)
	(3,1)
	(3,2)
	(4,-2)
	(4,-1)
	(4,0)
	(4,1)
	(5,-2)
	(5,-1)
	(5,0)
};
\addplot[white!70!black] fill between[of=A and B];
\end{axis}
\end{tikzpicture}
\noindent\begin{tikzpicture}
\begin{axis}[width = 4.5cm, ytick = {-2,-1,0,1,2}, xtick = {1,2,3,4,5}, xmin = 0.5, xmax = 5.500000, ymin = -2.5, ymax = 2.5, axis line style = {draw = none}, xmajorgrids, ymajorgrids, separate axis lines, yticklabels = {vb, b, n, g, vg}, xticklabels = {$c_1$,$c_2$,$c_3$,$c_4$,$c_5$}]
\addplot[name path = A, black, no markers, solid, very thick]
coordinates {
	(1,0)
	(2,0)
	(3,-1)
	(4,1)
	(5,0)
};
\addplot[name path = B, black, no markers, solid, very thick]
coordinates {
	(1,-2)
	(2,-2)
	(3,-2)
	(4,-2)
	(5,-2)
};
\addplot[name path = C, black, mark = *, only marks, very thick]
coordinates {
	(1,-2)
	(1,-1)
	(1,0)
	(2,-2)
	(2,-1)
	(2,0)
	(3,-2)
	(3,-1)
	(4,-2)
	(4,-1)
	(4,0)
	(4,1)
	(5,-2)
	(5,-1)
	(5,0)
};
\addplot[white!70!black] fill between[of=A and B];
\end{axis}
\end{tikzpicture}
	
		\vspace{1ex}
		\hrule
	
    \caption{Adjusted assignment rules for the final model of \textsc{CM2};}\label{fig:ex2-rules2}
\end{figure}