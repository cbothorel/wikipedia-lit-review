One of the most important task for the leaders or the animators of
virtual communities is to involve people and to keep them involved,
according to the literature on virtual management \citep[see][p. 15]{CrowstonHeckmanMisiolek10}.
According to \citet{ZhuKrautKittur12}, in Wikipedia too, leadership
behaviors matter to motivate people to participate, especially when
''transactional leadership and person-focused leadership'' are used.
Illustrating this point, \citet{BillingsWatts10} showed that thanks
to the wiki tool which allows citations, conciliators solve disputes
by helping differentiate ''the personal and substantive'' (p. 6)
in close interaction with the disputants, and \citet{Musicantetal11}
found ''significant correlations between editor communication and
article editing activity'' (but without being able to evaluate the
direction of the correlation).

This, however, seems to be insufficient to avoid the difficulties
the projects meet to keep the contributors we mentioned in the introduction.
\citet{Halfakeretal09} found that ''as they gain experience, contributors
are even more likely to have their work rejected''. As they explained
(p. 7), referring to \citet{Bryantetal05}, this may be due to the
fact that ''editors {[}are{]} being more bold as they gain experience'',
and \citet{HalfakerKitturRiedl11} found a positive impact of the
revert on the quality of the production. But they also showed that
the revert effect is particularly discouraging and excluding for newcomers.
The authors plead for a better communication toward these newcomers
specifically and the reverted in general, but the ones who are involving
in a discussion are more likely to stay and to produce work of better
quality. And \citeauthor{Musicantetal11}'s analysis of the Wikipedia
programs ''adopt a user'' (for the English Wikipedia) and ''Mentorenprogramm''
(for the German one) exhibited mitigate success, as ''communications
specifically between adopters and adoptees do not seem to offer advantages
over other forms of editor communication'' (2011, p. 9).

These meta-analyses, as the analyses on the production of an article
we presented before, show that the lens should be focused on a closer-interaction
management, at group and article level, where the day-to-day management
is conducted \citep{ForteLarcoBruckman09}. As \citet[table 1, p. 2]{Forteetal12}
pointed out, relying on \citeauthor{McGrath91}'s typology of (small)
group modes and functions (1991), this is exactly what these nested
organizations are made for, in addition to production activities support:
maintaining group's well being and providing support to members. Using
the concept of Legitimate Peripheral Participation, proposed by \citet{LaveWenger91},
in the case of Wikipedia, \citet{Bryantetal05} showed that the technical
structure of the project also facilitates newcomers' slow involvement
(they showed how the interface help to choose the task to do, or how
the article can be improved, which is also done at project level).
\citet{Choietal10}, in their examination of Wiki Projects showed
positive impact of welcome messages, assistance, and constructive
criticism on newcomers' edit levels. This finding is also consistent
with \citeauthor{Musicantetal11}'s recommendation to improve the
mentoring program by better matching mentor and adoptees on their
field of interest, to improve the ''empowerment mechanisms'' \citep{HansenBerenteLyytinen09}
of this project.

Two levels of leadership would then exist in Wikipedia. A project
leadership, focused on content, where discussion and coordination
are very linked to contribution at article level \citep{UngDalle10},
with strong socialization effects \citep{Forteetal12}, and a more
global management, aiming at solving the cases unsolved (in that respect,
\citealp{ZhuKrautKittur12} showed that what they called the ''legitimate
leaders'', or leaders having official responsibilities are more likely
to be followed). Here again, an analysis of people's comparison of
relative status \citep{Burt09,LeskovecHuttenlocherKleinberg10b} may
shed light on the process of interaction.

This two level of management is well illustrated by \citeauthor{Zhuetal11}'s
study (2011) on the differences between administrative persons (''admin''
or ''sysop'') and project leaders, in the English Wikipedia. They
developed a tool making possible to automatically assess the kind
of message sent (positive, negative, directive or social). If there
is no significant differences in the volume of messages posted by
these two types of leaders, they showed that local project leaders
leave more task oriented messages when administrators are more in
the social exchange, sending more personal messages (p. 4). This may
explain the difficulties for retaining newcomers, as the project leaders
would be too directive and not socializing enough. It can also prove
that the administrators intervene when people's behavior become a
threat to the well-being of the global project. But this study was
done on the personal pages' messages, thus on the most involved people's
exchanges, and skipped the project articles pages, where more social
exchanges may occur. So, as the authors acknowledged, if it gives
additional proves of the existence of those two levels of leadership,
it must be completed.

The admin election process gives insights of how these two levels
articulate. \citet{Ortega07} showed that the admins are not the ones
who contribute the more to the articles, and \citet{BurkeKraut08}
extended this point showing that the candidate\textquoteright s article
edits were weak predictors of success: they have to demonstrate also
managerial behaviors. Diverse experiences and contributions to the
development of policies and to WikiProjects are stronger predictors
of RfA {[}Request for Adminship{]} success. Future admins also use
article talk pages and comments for coordination and negotiation more
often than unsuccessful nominees, and tend to escalate disputes less
often. In addition to this, \citet{LeskovecHuttenlocherKleinberg10}
showed that the voters favor people who have the same characteristics
than them, i.e. who are on comparable or superior merit (and vote
negatively for those who are of lower merit, \citealt{LeskovecHuttenlocherKleinberg10b}),
especially when these people are in minority. \citet{CabunducanCastilloLee11}
showed that ''voters tend to participate in elections that their
contacts have participated in'' and that ''candidates who gain the
support of an influential coalition tend to succeed in elections''. 

To be exhaustive, a third level of leadership should be considered,
the Wikimedia foundation, and the process of election of the members
designed by the participants, studied. But the results already found
are consistent with the theoretical study made on leadership in self-managing
virtual teams by \citet{CrowstonHeckmanMisiolek10}, and based on
leadership theory and structuralism theory. According to these authors,
in these communities, there would be a ''first-order leadership {[}...{]}
that works within and reinforces existing structures to elicit and
guide group contributions'' and a ''second-order leadership as behavior
that effects changes in the structure that guides group action {[}...{]}
enabled by first-order leadership, therefore action embedded, and
grounded in processes that define the social identity of the team.
{[}...{]} effective self-managing virtual teams will exhibit a paradoxical
combination of shared, distributed first-order leadership complemented
by strong, concentrated, and centralized second-order leadership {[}...{]}''
(pp. 28-29).

Finally, this organization, and the evolution of the participants
from personal interest regarding the contribution toward collective
interest, but also the importance of the discussion about the building
of trust to become a regular contributor, echo reflexions on management,
especially stewardship practices \citep{DavisSchoormanDonaldson97},
and the alignment of people's interest and collective's interest \citep{Hernandez12}.
This leads us to a more general conclusion on the process.