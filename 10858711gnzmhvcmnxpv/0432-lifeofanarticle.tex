We will start with two main moments in an article life, the decisions
of deletion and of promotion, before looking at a larger-in-time process,
the cooperation around the article.

\paragraph{The decisions regarding an article.}

The editing arguments leading to these deletions are quite on line
with the rules of the project, as shown by \citet{WestLee11}, on
a corpus of one year deletions in the English Wikipedia: the non-respect
of the non-novelty rule, but also, the ones which ''present a legal
liability to the host (e.g., copyright issues, defamation), the harder
to detect, or privacy threats to individuals (i.e., contact information).''
If copyright issues are hard to discuss, the others are subject to
interpretation and the decision of deletion can be taken after a vote.
On that aspect, \citet{TaraborelliCiampaglia10} showed that there
is a ''herd effect'': ''an over- or under-expression of preferences
in the initial part results in an over- or under-expression in the
following'' (p. 3), which, according to the authors, can be due to
recruitment activities among voters' group, or as studied by \citet{GeigerFord11}
on the English Wikipedia, that the final decision remains to experimented
users, and that this expertise is recognized by the voters. \citet{LamKarimRiedl10}
analyzed the various elements impacting the quality of the deleting
decision and we reproduce their findings in table \ref{tab:Impact-of-structure-group},
page \pageref{tab:Impact-of-structure-group}.

% process of promotion of articles.

This mechanism is also apparent in the process of promotion of the
articles. \citet{KeeganGergle10}, who studied a corpus of 161 deliberations
(in a 3 month time frame), concluded that elite users ''fulfill a
unique gatekeeping role that permits them to leverage their community
position to block the promotion of inappropriate items. However, these
elite users are unable to promote their supported news items more
effectively than other types of editors.'' (on breaking news stories)

\begin{table}
\caption{\label{tab:Impact-of-structure-group}Impact of the structure of the
group on the quality of the decision regarding article deletion, from
\citet{LamKarimRiedl10}, table 3, p. 10.}

\begin{tabular}{|c|c|>{\centering}p{8cm}|}
\hline 
Hypothesis & Result & Description\tabularnewline
\hline 
\hline 
H1 Bigger-Better & Supported & Larger groups make better decisions, but with diminishing returns\tabularnewline
\hline 
H2 Recruit-Worse & Mixed & Biased recruitment leads to worse decisions under some circumstances\tabularnewline
\hline 
H3a Newcomers-Worse & Supported & Newcomer participation yields worse decision quality\tabularnewline
\hline 
H3b Diversity-Moderate & Mixed  & Diverse groups may make better decisions; no social categorization
effects were observed\tabularnewline
\hline 
H4 Biased-Admin-Worse  & Supported & Worse decisions in some cases if decision agrees with administrator\textquoteright s
bias\tabularnewline
 &  & Better decisions in some cases if decision is contrary to administrator\textquoteright s
bias\tabularnewline
\hline 
\end{tabular}
\end{table}

\paragraph{Conflict during the reaction of the articles.}

As \citet{GoldspinkEdmondsGilbert10} showed, the appeal to the rules
is rare in the discussion, in mean, but more important in the controversial
articles, as ''behavior seems to accord to a convention which editors
quickly recognize and conform to (or bring to the Wikipedia) and which
minimally accommodates what needs to be done to satisfy the task in
a context of divergent personal goals''. However, as pointed by \citet{Blacketal11},
who analyzed discussions and deliberation in small groups (based on
the deliberative discussion theory they developed in \citet{GastilBlack08}),
if the discussion groups present a ''relatively high level of problem
analysis and providing of information'', the ''results were mixed
in the group's demonstration of respect, consideration, and mutual
comprehension''. An hypothesis which has to be tested is that the
appeal to the rules is thus used only in the extreme cases when the
conflict cannot be solved, when personalities are involved, or when
points of views refer to different level of legitimacy.

Looking at the discussion pages and at the disputes allows to go deeper
in the discussion process around the creation of articles (see the
in-depth analysis of the conflicts in the French Wikipedia by \citealt{Aurayetal09}),
but also to discuss this hypothesis. The analysis of the conflict
by \citet{Kripleanetal07} showed that if the appeal to the policies
(the rules) is the main tool to resolve conflict, ''ambiguities in
policies give rise to power plays'' (how groups of contributors claim
legitimate control over content through the discourse of policy)\footnote{Even if Wikipedia has a rule, or a guide, about dispute resolution,
see \url{http://en.wikipedia.org/wiki/Wikipedia:Dispute_resolution}.}, and \citet{Nagar12} explained that the question-answer structure
forced by the wiki environment, and especially the talk pages, ''amplifies
the publicity and irrevocableness of volitional interacts, and thus
intensifies the process of turning them to commitments'' (p. 400),
facilitating the convergence.

Going deeper in the characteristics of the interaction and of the
sense building, \citet{Freardetal10} used a social role analysis\footnote{They relied on manual analysis of the conflict, but also on a tool
which automates these study via natural language analysis (in French).
This method, quite promising, may lead to new characterization of
the articles, in terms of level of conflict and of data produced,
and could complete \citet{FongBiuk-Aghai10}'s and \citet{ViegasWattenbergDave04}'s
works, in addition to \citeauthor{Kitturetal07a}'s mechanism to detect
conflicts (p. 6, the most important input, according to their model,
is the number of revision).} to study how different contributors participate to the construction
of an article taken as collective output, here on the conflict on
Pluto article discussion page (is or is not Pluto a planet?) They
spotlight the debate/conflict between academic knowledge (Pluto is
not a planet because it has been defined as not being a planet) and
Wikipedia structure (people will look for Pluto as a planet and they
have to find it like that, even if later the article tells them they
are wrong), which can lead to personal conflict. It is interesting
to note that when personal conflict happens, the discussion and the
reverts may discourage people to participate to the article, as, as
shown by \citet{Kitturetal07a}, ''the number of unique editors involved
in an article negatively correlates with conflict to be read'' (p.
6). It would be interesting to extend these studies looking at the
relative status of the persons involved in the conflict to see if,
more than a formal position, people look at relative position to accept
or refuse each other proposals, as recent works in social networks
indicated \citep{Burt09,LeskovecHuttenlocherKleinberg10b}.

When these conflicts are unsolvable, they can lead to the exclusion
of a participant, or to the move of the conflict to another space
(appeal to a mediator, or, as \citet{BillingsWatts10} called them,
a ''conciliator'', moving of the dispute in the conciliation section
of the site, where disputants are isolated from the others with the
moderator). This provides the organization with multi-levels way of
coping with conflicts, before reaching the extreme measure of blocking
this conflict, at article level (freezing an article) or at individual
level (excluding a person). In both cases, a managerial decision is
taken, decision we are looking at in the next paragraph.