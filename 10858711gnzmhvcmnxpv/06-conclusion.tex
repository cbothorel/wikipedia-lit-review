ost of the analyses we discussed used dump data because they are
rather complete on the contributors the most involved, especially
on the process part. But, as \citet{PreeceShneiderman09} claimed,
if more studied are still needed to understand the Reader-to-leader
process, these analyses may be completed with more qualitative studies
and surveys. We found very few works, and this one does not help for
that, on the non-text production collaboration, such as the one by
\citet{Viegas07} on the images producers. However, it is worth noting
that the results founds on the reasons to participate, the structure
of interaction and of governance seem close to the vision administrators
have from the inside of the Encyclopedia, \citep[See][on the interview of Swedish Wikipedia Administrators]{Mattus09}.

These results are also coherent with \citet{OstromHess06}'s framework
and description of the knowledge commons: if finding the community
is easier than other communities of practice (see, for instance, \citealp{MerriamCourtenayBaumgartner03}
on the access to a community of practice of which), there is a period
of apprenticeship, to do so, people are nested in small groups, dedicated
to topics they were concerned about since the beginning of their participation.
Some rules structure this community, but are constantly under discussion
and constantly evolving to adapt the project to its environment and
its participants.

As announced at the beginning of this article, we did not look at
the feedback loop of \citeauthor{HessOstrom06b}'s model, the impact
of Wikipedia on its environment. There are a very actual and active
discussions in the librarian and the teaching communities on how integrating
the encyclopedia in their professional practices. If we only tangented
these debates in this work, especially when looking at Wikipedia quality,
we hope that this article may help the discussants to better understand
how the project works. The discussion by \citet{Konieczny09b} of
Wikipedia being or not a ''social movement'' participates to the
same questioning about the impact of this collective action on the
society, on the way the knowledge is produced and transmitted, but
also to a broader discussion of the links between the contribution
to online open communities and the involvement of a job (on that respect,
see the discussion proposed by \citet{Brown08}, relying on \citet{Himanen02},
on open-source and Wikipedia participants, where he argued that these
hackers are creating new borders between work and leisure).

% What is Wikipedia, impact on society

Finally, being such a successful collective action of creation of
a common, Wikipedia has been taken as an emblem of the wisdom of the
crowd \citep{Surowiecki04}. It is a system, even if an imperfect
one, which allows more to produce and discuss knowledge, thus, in
that sense, empowering more than traditional encyclopedias \citep{HansenBerenteLyytinen09}.
This does not mean either that everybody has access to the process,
as the technological boundaries, but also the organizational ones
remain important, a pointed out by \citet{Hartelius10} and \citet{Pfister11}
as by \citet{Perovic09}. Does that mean, as these two authors debate,
that Wikipedia is a thinking machine which overcomes the human fallibility
thanks to a technical system, and thus institutes a socio-technical
expertise instead of the traditional scientific expertise, and a as
argued by \citet{Perovic09}, a world ''of contingency without irony,
knowledge without self-observation and learning without thinking,
a world enshrined by Wikipedia today.''? Does this ''flawed knowledge
communities'' \citep{RobertsPeters11}, which is always discussing
its perimeter, as \citet{Kostakis10} showed in his analysis of inclusionists
and delationists, or \citet{deLaat12} on his analysis of the rules
regulating new edition, the future deposit of the human knowledge?

In addition to the doubtful argument that a system of production may
replace another, comparing Wikipedia with traditional encyclopedias
may simply miss the point. As explained by \citet{Mattus09}: it has
to be seen as one entry, always evolving to access to knowledge, but
which may be combined with others (scientific references, traditional
encyclopedias), and used as a tool amongst others, and not, as was
the Encyclopédie, the deposit of the human knowledge. What Wikipedia
shows is the extension of the knowledge and of the sources of knowledge,
since the seventeenth century, and thus the never ending need to educate
the users to have a critical, scientific reading of any source of
knowledge.
