\citet{Voss05} provided with the first global figures on the articles
(mostly on the German Wikipedia), showing a lognormal-like distribution
in their size (ibid, Figure 3, p. 6), which stabilizes when the project
gets a certain size (even if the articles' mean size is growing) and
that the number of distinct authors per article follows a power law
(Figure 4, p. 7), as the numbers of ingoing and outgoing links (ibid,
Figure 7, p. 9). \citet{NazirTakeda08} found globally the same results
on the English Wikipedia for the number of users per article. \citet{Capoccietal06}
looked at the links between articles and found that the structure
of these links is closed to the World Wide Web's one and found ''a
scale-invariant distribution of the in and out degree''. This led
them to conclude that ''Wikipedia growth can be described by local
rules such as the preferential attachment mechanism\footnote{From \citet{BarabasiAlbert99}, this mechanism says that the more
popular articles (in terms of contributors, links...) are, the more
likely to increase their popularity is. }, though users, who are responsible of its evolution, can act globally
on the network.'' Note that this preferential attachment mechanism
has been proposed as an explanation by \citet[p. 9]{Voss05}. These
considerations have been summarized by \citet{WangJinWu09}, saying
that a small number of people is strongly connected to lots of people
and assures the coherence and the small world effect of the model,
whereas the vast majority aggregates around centers of interests and
is poorly connected. Also, the more popular topics (and central people)
are those who aggregate new people the more (\citealp{KeeganGergleContractor12}
gave an example of this phenomenon: looking at articles about plane
accidents, they showed that breaking news articles are those which
are the most likely to attract new editors, when experienced users
remained more focus on their own agenda and editing 'their' set of
article).

\citet{Zlaticetat06} did similar analyses on the main Wikipedia languages
projects and found that ''degree distributions, growth, topology,
reciprocity, clustering, assortativity, path lengths'' are common,
and defended the idea of an unique growth process. Finally, \citet{WangMaCheng10},
studying the categories (or tags) which define the pages, showed that
there is an obsolescence phenomenon, as if the categories followed
the actuality or that they are created when the articles and the topics
they refer to are created. 

One has to be very careful, however, to draw general conclusions from
the analysis of the global structure of the project. As shown by \citet{Silvaetal11},
who analyzed four sub-projects of the English Wikipedia (Biology,
Mathematics, Physics and Medicine). These sub-projects present very
different structures regarding the links between articles, with dense
links in biology and medicine and less in physics and Mathematics.
The reasons for these differences are not explicit, and more work
would be useful to understand if they are due to the internal characteristics
of the disciplines or to the internal organization of the Wikipedia
sub-projects. (We will come back to this point in the discussion on
the growth of the project at the end of this part).

We have not found any comparison with the other online encyclopedias
regarding the global structure (distribution of the size and of the
links of an article), a study which would be interesting to conduct
to see if Wikipedia is different on this aspect.