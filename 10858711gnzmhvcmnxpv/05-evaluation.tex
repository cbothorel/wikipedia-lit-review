As explained by \citet[p. 3]{Hammwohner07}, Wikipedia quality can
be assessed either via internal measures (Wikipedia has several levels
of quality for its articles, from article needing to be improved to
featured article, see \url{http://en.Wikipedia.org/wiki/Wikipedia:Article_development}),
as in \citet{Poderi09,CarilloOkoli11}; either by external metrics
coming from both the information system studies on the quality of
the process, on the information system (\citealp{Stviliaetal08},
\citealp{WohnerPeters09}), and the bibliography studies (on the product);
or subjectives metrics (user experience, being reader or producer).
As pointed out by \citet[p. 6-7]{LewandowskiSpree11}, referring to
\citet{Geeb98}, the evaluation of the quality of a work can not be
done independently from the definition of the user: ''degree of expertise
such as layperson or expert, user situation referring to the actual
usage such as text production or understanding, and user intention,
which can widely vary from gathering factual information to background
information or references''. And, in a way, as Wikipedia addresses
both specialists and novices, learners and information checkers, an
analysis should be done for each of these users and uses. However,
some indicators exist to evaluate the quality of an article, or the
coverage of a project, and these indicators have been used for Wikipedia
too. Considering this, and following the main parts of \citet{LewandowskiSpree11},
not in the same order, we will look at the quality of the process,
the user experience and the external evaluations of the quality of
Wikpedia's entries (articles) and coverage.