\paragraph{System use.}

One of the obvious way to measure the users' interest for the encyclopedia
is to measure the frequenting, in total or regarding the topics. Various
tools to do so are available and presented on a dedicated Wikipedia
page (\url{http://en.wikipedia.org/wiki/Wikipedia:Statistics} ).
The Wikipedia Web site is ranked 6 in the world, according to Alexa\footnote{\url{http://www.alexa.com/siteinfo/wikipedia.org+sina.com.cn+orkut.com+live.com+youtube.com}}.
The availability of these statistic tools may explain that there are
few researches on the frequenting, or because, as pointed by \citet{Spoerri07a},
despite all these tools, there is no will from the site to present
the most visited article, like in Youtube or Digg\footnote{\url{http://www.youtube.com/browse?s=mp&t=m&c=0&l=} and \url{http://digg.com/news/month}}. 

\citet[a, b]{Spoerri07a}, \nocite{Spoerri07b} looked at the 100
most visited pages, and at their stability during several following
months. He showed that the encyclopedia answers to two kind of researches,
on the actuality (death, new films), which vary over time, and on
major historical events (World War 2, for instance), which are more
stable. If the vast category ''Entertainment'' represents 43\% of
the articles in the top 100 (and sexuality 10\%), Politics and History
(15\%), Geography (12\%) and Science (6\%) represent major categories
for the visits. As pointed out by the author, Wikipedia is primarily
accessed via the browsers and this list reflects the research on the
Web for which Wikipedia is one site of reference, thus the actuality,
but also some more stable interests among the people (as shown by
the analysis by \citealp{Ratkiewiczetal10}). In that aspect, Wikipedia
can be seen as the encyclopedia of the everyday life, as had been
the French Encyclopedia ''Quid''\footnote{\url{http://fr.wikipedia.org/wiki/Quid}}
during 40 years, before being killed by the online encyclopedias \citep{CapitalQuid07}.
But it is also an encyclopedia of reference for researchers, integrated
in their routines when dealing with peer-reviewed research report
\citep{Dooley10}. Finally, the confidence in the self capacity to
evaluate the accuracy of the information is important in the users'
experience, as shown by \citet{LimKwon10} in their survey of undergraduate
students. They also showed gender differences, such as the fact that
males use more Wikipedia for entertainment and idle reading than females,
and have better expectation and reward from using this tool. 

Finally, and as already mentioned in the introduction, the evaluation
of Wikipedia as a credible source of information has been debated
since its origin, in press, and in scientific articles. This leads
today to a bias perception of its quality by the users \citep{FlanaginMetzger11}:
''children rated information from Wikipedia to be less believable
when they viewed it on Wikipedia's site than when that same information
appeared on either Citizendium's site or on Encyclopædia Britannica's
site''. The analyses in terms of external measure or quality or in
terms of user experience are more nuances on this aspect of quality.
In his survey of 50 academics on Wikipedia articles in their area
of expertise against articles outside it, \citet{Chesney06} showed
that the experts (people who evaluate an article in their area of
expertise) found it more accurate than people who evaluate an article
outside their area of expertise.

\paragraph{Accessibility.}

A way to improve user experience is to improve, if possible, article
accessibility, and the navigation in the site. \citet{LopesCarrico08}
proposed a global model for evaluation accessibility to major websites,
which have, such as Wikipedia, different types of users, aiming at
accessing via different types of devices for different intentions,
in different usage situation, but they did not do an analysis of Wikipedia.
Some problems remain, summarized in the page dedicated to this point
by the MediaWiki project\footnote{See the page dedicated to this particular problem: \url{http://blind.wikia.com/wiki/Mediawiki_and_Accessibility}},
but they do not seem to have attract a lot of researchers for the
time being.

When an evaluation is done on the accessibility of the article, such
as the one done by \citet{MuhlhauserOser08} on medical articles,
it shows that these articles are viewed as less understandable than
other sources (here major health insurance Web site), by the evaluators
(med school students) but, as recognized by the authors, the study
is very partial.