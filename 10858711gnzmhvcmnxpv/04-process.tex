On the contrary, Wikipedia allows to access to a complete set of data
about the articles, their evolution, the people who contributed to
them, but also to the discussions which occurred before, during and
after the contributions. The articles exploring and exploiting this
fascinating set of data to better understand how people interact in
such an information system to product a public knowledge are mainly
threefold: first, the articles focusing on people and on their actions,
describing the activity and assessing roles from this activity; second,
the articles looking at the ''product'', the article; and third
the process of creation of such pieces of knowledge, with some works
looking at the other pieces of knowledge created in Wikipedia (mainly
the discussion pages). In general, the studies looking at the global
organizational structure rely on statistical analyses of the variables
present in the databases (Dump). Following the seminal work of \citet{KorfiatisPoulosBokos06},
most of these studies use social network analysis techniques, the
nodes being, usually, the people, and the arcs the fact they contribute
to the same article or the same talk page. On the article side, the
node are the articles and the arcs the fact that they refer to each
others (sometimes those two approaches are mixed). When seeking to
improve the processes of collaboration, scholars privileged usually
more narrowed sets of articles, their evolution and the one of their
talk pages, but deepening the analyses of the content produced.

\citet{Slattery09} used a quite similar segmentation and provides
a nice first approach to the main characteristics of these patterns,
approach we are developing here.