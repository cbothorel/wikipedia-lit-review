\citet[p. 18]{KitturKraut08} noted that \textquotedblleft decades
of research in organizations show that communication as the basis
for coordination is especially important in tasks that are highly
uncertain, unconstrained, and subject to many changes\textquotedblright ,
as can be the construction of the content of an article (\citet{CardonLevrel09,Cardon12}
explained, in the particular case of Wikipedia, why and how the rules
in use are not always enough). This explains the importance of the
talk pages in the project, even if there are variations between the
projects (\citealp{Voss05} showed strong differences in the ration:
user talk over user pages, between the European projects (German 0.94,
Danish 0.88, Croatian 0.74) and the Japanese one, 2.51). \citet{PancieraHalfakerTerveen09},
quoting \citeauthor{Viegasetal07}'s result that over half of Talk
page comments are requests for coordination and 8\% are policy invocations
(2007), concluded that Wikipedia contains strong and supportive communities.
\citet{Butleretal08} also found that much of the explicit coordination
is managed through the Talk or discussion pages for the article in
question.

Not surprisingly, if taking into account Actor Network Theory \citep{Latour05,AkrichCallonLatour06},
studying these discussions, and especially the ''conflicts'', can
give explanations on the repartition of the work between direct contribution
to articles, vandalism fighting and discussion or negotiation of the
''point of view'', leading to mutual adjustments when the rules
are not enough to do so, but also of the brutal rejection of ''pathologic''
discussants \citep{Aurayetal09}. It is also a well know issue for
distant (virtual) organizations \citep{HindsBailey03,HindsMortensen05},
both negative because it consumes people's time, and positive because
it can strengthen the community \citep{Francoetal95}\footnote{For a discussion of the cause of conflict, the way the Wikipedia organization
could avoid them and the need for a better understanding of the process
of conflict management, see the study of ''the bibliography of the
living persons in Wikipedia'' by \citet{JoyceButlerPike11}.}. We have not found any survey on the patterns of interaction, probably
because they will suffer from the same problem as the ones regarding
implication, and because of the already rich data available. These
surveys would be of help, however, to understand how people choose
the article they contribute to (topics, people working on it), and
their perception of the conflicts, and thus to deeper the understanding
of the structure of interaction the data analyses make apparent.

But before looking, thanks to the discussion pages (and the analyses
made on them), at the processes behind the life of an article (creation,
deletion, evolution, promotion), and at the managerial behaviors,
and to follow up the discussion started, we will start giving some
results on the who and the how of the construction of an article.
In a word, what a 'good' team to construct an article is.