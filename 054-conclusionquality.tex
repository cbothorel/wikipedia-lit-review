These analyses are not contradictory. First of all, Wikipedia is a
project still young and improving: \citet{Nielsen07} showed an increase
over time of the quality of the citations and of the references to
scientific journals. Secondly, and coming back to user experience,
one has to be careful in comparisons: as pointed out by \citet{Fallis08},
the quality of Wikipedia should not be evaluated in comparison with
non accessible encyclopedia, but taking into account the way people
use information online (is Wikipedia better than no-Wikipedia). The
fact that Wikipedia is less trustworthy than, say Encyclopedia Britannica,
seems to be internalized by people (at least faculties), as for the
same content, an article branded Wikipedia is view as less accurate
that when it is branded Encyclopedia Britannica \citep{KubiszewskiNoordewierCostanza11}.
In addition to this, Wikipedia is quite irreplaceable for accessing
information about the fact and figures in entertainment, but also
in the scientific actuality. 

It is a project with multi-dimensions, trying to keep memory of breaking
news, people, as well as scientific facts and concepts and quite rooted
in the actuality. The quality of the articles varies thus also according
to external events and is decreasing with the age of the event: \citet{Hjorland10}
showed that the English Wikipedia is the best source when dealing
with the research of information on a controversial topic (here breast
cancer screening) and, according to \citeauthor{Brown11}'s study
of political subject (2011), however almost always accurate when the
article exists, Wikipedia is better for actual topics than for ''older
or more obscure subjects''. It is also a (one of the?) place(s) where
the importance of events is negotiated, leading \citet{Pentzold09}
and \citet{HaiderSundin10} to describe Wikipedia as a discursive
fabric of collective memory. One of the possible drawbacks of this,
as pointed \citet{Elvebakk08} in his study of articles on philosophers
in Wikipedia and two other sources, from the viewpoint of the academic
discipline, is the fact that having much more people having an article
in Wikipedia, could lead to over-coverage and poor signal on the quality
of the persons covered.

In a word, if it is a good point of entry (or at least not worse than
another) for specialized knowledge, it should be competed to get accurate
information, which was already \citeauthor{Clausonetal08}'s conclusion
for medicine (2008) and \citeauthor{Korosec10}'s for chemical information
(2010). This is, actually, exactly what Wikipedia says about itself,
as pointed by \citet[p. 596]{Murley08}. This what leads \citet{Haigh11},
assessing the quality studying the references founds in Wikipedia's
articles to conclude that, regarding the high rate of references of
identifiable source, Wikipedia is appropriate for use by nurse students
(if they look at the references, which \citet{Korosec10} showed to
be far from obvious). And \citet{Devganetal07} concluded that, given
the popularity of the Web site, the medical and surgical professionals
should improve the weakest entries (\citealp{WestWilliamson09}, propose
advices to do, especially when trying to involve students).
